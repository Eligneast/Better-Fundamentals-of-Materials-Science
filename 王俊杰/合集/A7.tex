\chapter{超声波无损探伤实验}
\section{实验目的}
    \begin{enumerate}
        \item 学习超声波的产生原理及特点,了解超声波探伤仪的工作原理及使用方法 
        \item 学习超声探头指向性原理及其实际应用
        \item 认识超声换能器及超声波探头的不同种类
        \item 了解超声波的传播、波型和波型转换原理及超声波声速的测量方法
    \end{enumerate}
\section{实验原理}%简单描述,含必要的公式和附图;
    超声波是频率高于 20kHz 的机械波,具有穿透力强、传播方向性好等特点,实验所用探头通过压电晶片的逆压电效应,将电能转为机械能以产生脉冲超声波。超声波探头结构复杂,包括压电晶片、保护膜、匹配电感等部分,根据不同结构和应用情况可分为直探头、斜探头和可变角探头。超声波的指向性是指探头发射的声束扩散角大小,与波长、频率及探头内压电晶片尺寸有关。声束扩散角越小,指向性越好,测量精度越高。指向性大小可用公式表示:
    \begin{equation}
        \theta=2\sin^{-1}\left(1.22\frac\lambda D\right)
    \end{equation}\par
    超声波根据波形可分为超声纵波、超声横波和超声表面波三种,在介质界面发生反射、折射和波形转换时需满足斯特令定律:
    \begin{equation}
        \begin{aligned}
            \text{反射:}\frac{\sin\alpha}C&=\frac{\sin\alpha_L}{C_{1L}}=\frac{\sin\alpha_S}{C_{1S}}\\
            \text{折射:}\frac{\sin\alpha}C&=\frac{\sin\beta_L}{C_{2L}}=\frac{\sin\beta_S}{C_{2S}}
        \end{aligned}
    \end{equation}\par
    测量超声波声速可采用直接或相对测量法。直接测量法利用超声波探头内部延迟时间和探头测量的人工反射体回波时间计算声速;相对测量法则通过测量两次反射回波的时间差来计算声速。
\section{实验仪器}%规格及参数
    JDUT-2 型超声波试验仪、DS1102E 双通示波器(100MHz)、直探头、斜探头、CSK-IB 试块、耦合剂(水)等。
\section{实验过程}%简述主要过程和实验内容
    \begin{enumerate}
        \item 组装实验仪器;
        \item 利用直探头测量脉冲超声纵波频率和波长
        \item 测量直探头延迟和间接测量法测量试块纵波声速
        \item 测量斜探头延迟、入射点和间接测量法测试块横波声速
        \item 测量斜探头的折射角
        \item 测量直探头和斜探头的声束扩散角
        \item 使用直探头探测缺陷深度
        \item 使用斜探头探测待测试块内部缺陷位置
    \end{enumerate}
\section{实验数据与分析}
\subsection{实验数据\label{sss:data}}
\begin{table}[!ht]
    \caption{CSK-IB 铝试块参数测量}\label{tab:A7.length}
    \begin{tabularx}{\textwidth}{*{7}{C}} \toprule
        测量量 & 1 & 2 & 3 & 4 & 5 & 平均值 \\ \midrule
        $L_A$(\unit{\mm}) & 19.8 & 19.9 & 19.8 & 19.8 & 19.9 & 19.84 \\ 
        $L_B$(\unit{\mm}) & 50.0 & 50.0 & 50.1 & 50.0 & 50.0 & 50.02 \\ 
        $L_{AB}$(\unit{\mm}) & 30.2 & 30.1 & 30.3 & 30.2 & 30.1 & 29.86 \\[2mm]
        $H_A$(\unit{\mm}) & 20.3 & 20.2 & 20.4 & 20.3 & 20.3 & 20.30 \\
        $H_B$(\unit{\mm}) & 50.4 & 50.4 & 50.3 & 50.3 & 50.4 & 50.36 \\
        $H_{AB}$(\unit{\mm}) & 30.1 & 30.2 & 29.9 & 30.0 & 30.1 & 30.06 \\ \bottomrule
    \end{tabularx}
\end{table}

\begin{table}[!ht]
    \caption{利用直探头测量脉冲超声纵波频率和波长}\label{tab:A7.exp1}
    \begin{tabularx}{\textwidth}{*{7}{C}} \toprule
        测量量 & 1 & 2 & 3 & 4 & 5 & 平均值 \\ \midrule
        $t$(\unit{\us}) & 1.50 & 1.47 & 1.49 & 1.50 & 1.49 & 1.49 \\ \bottomrule
    \end{tabularx}
\end{table}

\begin{table}[!ht]
    \caption{测量直探头延迟和间接测量法测量试块纵波声速}\label{tab:A7.exp2}
    \begin{tabularx}{\textwidth}{*{7}{C}} \toprule
        测量量 & 1 & 2 & 3 & 4 & 5 & 平均值 \\ \midrule
        $t_1$(\unit{\us}) & 15.26 & 15.24 & 15.23 & 15.22 & 15.24 & 15.238 \\ 
        $t_2$(\unit{\us}) & 29.71 & 29.68 & 29.67 & 29.66 & 29.67 & 29.678 \\ \bottomrule
    \end{tabularx}
\end{table}

\begin{table}[!ht]
    \caption{测量斜探头延迟、入射点和间接测量法测试块横波声速}\label{tab:A7.exp3}
    \begin{tabularx}{\textwidth}{*{7}{C}} \toprule
        测量量 & 1 & 2 & 3 & 4 & 5 & 平均值 \\ \midrule
        $t_3$(\unit{\us}) & 25.26 & 25.27 & 25.26 & 25.26 & 25.26 & 25.262 \\ 
        $t_4$(\unit{\us}) & 44.28 & 44.28 & 44.28 & 44.27 & 44.28 & 44.278 \\ \bottomrule
    \end{tabularx}
\end{table}

\begin{table}[!ht]
    \caption{测量斜探头入射点}\label{tab:A7.exp4}
    \begin{tabularx}{\textwidth}{*{7}{C}} \toprule
        测量量 & 1 & 2 & 3 & 4 & 5 & 平均值 \\ \midrule
        $L$(\unit{\mm}) & 42.5 & 42.3 & 42.2 & 42.3 & 42.2 & 42.3 \\ \bottomrule
    \end{tabularx}
\end{table}

\begin{table}[!ht]
    \caption{测量斜探头的折射角}\label{tab:A7.exp5}
    \begin{tabularx}{\textwidth}{*{7}{C}} \toprule
        测量量 & 1 & 2 & 3 & 4 & 5 & 平均值 \\ \midrule
        $L_{A1}$(\unit{\mm}) & 27.4 & 27.5 & 27.5 & 27.5 & 27.5 & 27.48 \\ 
        $L_{B1}$(\unit{\mm}) & 87.2 & 87.1 & 87.1 & 87.0 & 87.2 & 87.12 \\ \bottomrule
    \end{tabularx}
\end{table}

\begin{table}[!ht]
    \caption{测量直探头和斜探头的声束扩散角}\label{tab:A7.exp6}
    \begin{tabularx}{\textwidth}{*{7}{C}} \toprule
        测量量 & 1 & 2 & 3 & 4 & 5 & 平均值 \\ \midrule
        $x_1$(\unit{\mm}) & 44.2 & 44.1 & 44.4 & 44.3 & 44.2 & 44.24 \\ 
        $x_2$(\unit{\mm}) & 57.1 & 57.0 & 57.0 & 56.9 & 57.1 & 57.02 \\ 
        $x_3$(\unit{\mm}) & 82.9 & 83.0 & 83.0 & 82.9 & 83.0 & 82.96 \\ 
        $x_4$(\unit{\mm}) & 94.1 & 94.2 & 94.1 & 94.1 & 94.0 & 94.10 \\ \bottomrule
    \end{tabularx}
\end{table}

\begin{table}[!ht]
    \caption{使用直探头探测缺陷深度}\label{tab:A7.exp7}
    \begin{tabularx}{\textwidth}{*{7}{C}} \toprule
        测量量 & 1 & 2 & 3 & 4 & 5 & 平均值 \\ \midrule
        $t_C$(\unit{\us}) & 15.38 & 15.37 & 15.38 & 15.37 & 15.38 & 15.376 \\ \bottomrule
    \end{tabularx}
\end{table}

\begin{table}[!ht]
    \caption{使用斜探头探测待测试块内部缺陷位置}\label{tab:A7.exp8}
    \begin{tabularx}{\textwidth}{*{7}{C}} \toprule
        ~ & 1 & 2 & 3 & 4 & 5 & 平均值 \\ \midrule
        $t_A$(\unit{\us}) & 25.404 & 25.400 & 25.404 & 25.404 & 25.408 & 25.404~~{} \\
        $t_B$(\unit{\us}) & 51.548 & 51.552 & 51.552 & 51.548 & 51.556 & 51.5512 \\
        $t_D$(\unit{\us}) & 33.264 & 33.272 & 33.272 & 33.264 & 32.272 & 33.0688 \\[2mm]
        $x_A$(\unit{\mm}) & ~~28.6 & ~~29.0 & ~~29.0 & ~~28.9 & ~~29.0 & ~~28.90 \\
        $x_B$(\unit{\mm}) & ~~86.9 & ~~86.9 & ~~87.0 & ~~86.9 & ~~87.0 & ~~86.94 \\
        $x_D$(\unit{\mm}) & 107.0 & 106.9 & 107.0 & 107.1 & 107.1 & 107.02 \\
        \bottomrule
    \end{tabularx}
\end{table}
\clearpage
\subsection{间接测量量计算公式\label{sss:formula}}
\begin{align*}
    \text{超声波频率}\; f&=\frac 4 t \\
    \text{直探头延迟}\; t_s&=2t_1-t_2 \\
    \text{纵波声速}\; c_L&=\frac{2H}{t_2-t_1} \\
    \text{纵波波长}\; \lambda_L&=\frac{c_L}{f} \\
    \text{斜探头延迟}\; t_i&=2t_3-t_4 \\
    \text{横波声速}\; c_S&=\frac{2(R_2-R_1)}{t_4-t_3} \\
    \text{横波波长}\; \lambda_S&=\frac{c_S}{f} \\
    \text{斜探头前沿距离}\; L_0&=R_2-L \\
    \text{斜探头折射角}\; \beta&=\tan^{-1}\left(\frac{L_{B1}-L_{A1}-L_{AB}}{H_{AB}}\right) \\
    \text{直探头声束扩散角}\; \theta_s&=2\tan^{-1}\frac{\left|x_2-x_1\right|}{2H_B} \\
    \text{斜探头声束扩散角}\; \theta_i&=2\tan^{-1}\left(\frac{\left|x_4-x_3\right|}{2H_B}\cos^2 \beta \right) \\ 
    \text{缺陷深度}\; H_C&=c_L\frac{t_c-t_s}{2} \\
    \text{试块内超声声速}\; c&=\frac{2H_{AB}}{(t_B-t_A)\cos \beta} \\
    \text{探头延迟}\; t_0&=t_B-\frac{2H_{B}}{c\cos \beta} \\
    \text{前沿距离}\; L_0^{'}&=H_B\tan\beta-(x_B-L_B) \\
    \text{缺陷 D 的垂直深度}\; H_D&=\frac{c(t_D-t_0)\cos\beta}{2} \\
    \text{缺陷 D 的水平距离}\; L_D&=x_D+L_0^{'}-H_D\tan\beta
\end{align*}
\subsection{不确定度的计算}
利用 \ref{sss:data} 的数据以及\textbf{\nameref{sss:uncertaintyFormula}}与 \ref{sss:formula} 中的公式可求得各个测量量的不确定度和自由度,如表 \ref{tab:A7.uncertaintyResult} 所示,部分表头省略了不确定度四字。扩展因子是在置信概率为 95\% 的情况下由有效自由度计算得出。
\begin{longtable}{@{}>{\hfil}p{0.18\textwidth-1em}<{\hfil}@{\hspace*{1mm}}%
    S[table-format=3.6]S[table-format=2.7]S[table-format=2.6]%
    S[table-format=2.6]S S[table-format=2.6]}
    \caption{各测量量的不确定度\label{tab:A7.uncertaintyResult}} \\ \toprule
    测量量 & {平均值} & {A 类} & {B 类} & {合成} & {有效自由度} & {扩展} \\ \midrule
    \endfirsthead

    \multicolumn{7}{r}{\small 表 \ref{tab:A7.uncertaintyResult} (续)} \\ \toprule
    测量量 & {平均值} & {A 类} & {B 类} & {合成} & {有效自由度} & {扩展} \\ \midrule
    \endhead

    \bottomrule
    \endfoot

    \bottomrule
    \endlastfoot

    $L_A$(\unit{\mm}) & 19.84 & 0.0245 & 0.0577 & 0.0627 & 171.9 & 0.124 \\
    $L_B$(\unit{\mm}) & 50.02 & 0.02 & 0.0577 & 0.0611 & 348.4 & 0.12 \\
    $L_{AB}$(\unit{\mm}) & 30.18 & 0.0374 & 0.0577 & 0.0688 & 45.7 & 0.139 \\
    $H_A$(\unit{\mm}) & 20.3 & 0.0316 & 0.0577 & 0.0658 & 75.1 & 0.131 \\
    $H_B$(\unit{\mm}) & 50.36 & 0.0245 & 0.0577 & 0.0627 & 171.9 & 0.124 \\
    $H_{AB}$(\unit{\mm}) & 30.06 & 0.051 & 0.0577 & 0.077 & 20.8 & 0.16 \\[1em]

    $t$(\unit{\us}) & 1.49 & 0.00548 & 0.00577 & 0.00796 & 17.8 & 0.0167 \\
    $f$(\unit{\MHz}) & 2.68456 & 0.00987 & 0.0104 & 0.0143 & 17.8 & 0.0301 \\[1em]

    $t_1$(\unit{\us}) & 15.238 & 0.00663 & 0.00577 & 0.00879 & 12.4 & 0.0191 \\
    $t_2$(\unit{\us}) & 29.678 & 0.0086 & 0.00577 & 0.0104 & 8.4 & 0.0237 \\
    $t_s$(\unit{\us}) & 0.798 & 0.0158 & 0.0129 & 0.0204 & 19.1 & 0.0427 \\
    $c_L$(\unit{\mm\per\us}) & 6.23269 & 0.00469 & 0.00352 & 0.00587 & 18.4 & 0.0123 \\
    $\lambda_L$(\unit{\m}) & 2.32168 & 0.00871 & 0.00909 & 0.0126 & 18.9 & 0.0264 \\[1em]

    $t_3$(\unit{\us}) & 25.262 & 0.002 & 0.00577 & 0.00611 & 348.4 & 0.012 \\
    $t_4$(\unit{\us}) & 44.278 & 0.002 & 0.00577 & 0.00611 & 348.4 & 0.012 \\
    $t_i$(\unit{\us}) & 6.246 & 0.00447 & 0.0129 & 0.0137 & 512.4 & 0.0268 \\
    $c_S$(\unit{\mm\per\us}) & 3.15524 & 0.000469 & 0.00135 & 0.00143 & 696.9 & 0.00282 \\
    $\lambda_S$(\unit{\m}) & 1.17533 & 0.00432 & 0.00458 & 0.0063 & 18.1 & 0.0132 \\[1em]

    $L$(\unit{\mm}) & 42.3 & 0.0548 & 0.0577 & 0.0796 & 17.8 & 0.167 \\
    $L_0$(\unit{\mm}) & 17.7 & 0.0548 & 0.0577 & 0.0796 & 17.8 & 0.167 \\[1em]

    $L_{A1}$(\unit{\mm}) & 27.48 & 0.02 & 0.0577 & 0.0611 & 348.4 & 0.12 \\
    $L_{B1}$(\unit{\mm}) & 87.12 & 0.0374 & 0.0577 & 0.0688 & 45.7 & 0.139 \\
    $\beta$(${}^\circ$) & 44.4224 & 0.0734 & 0.112 & 0.134 & 138.5 & 0.264 \\[1em]

    $x_1$(\unit{\mm}) & 44.24 & 0.051 & 0.0577 & 0.077 & 20.8 & 0.16 \\
    $x_2$(\unit{\mm}) & 57.02 & 0.0374 & 0.0577 & 0.0688 & 45.7 & 0.139 \\
    $x_3$(\unit{\mm}) & 82.96 & 0.0245 & 0.0577 & 0.0627 & 171.9 & 0.124 \\
    $x_4$(\unit{\mm}) & 94.1 & 0.0316 & 0.0577 & 0.0658 & 75.1 & 0.131 \\
    $\theta_s$(${}^\circ$) & 14.4628 & 0.0712 & 0.0929 & 0.117 & 54.7 & 0.235 \\
    $\theta_i$(${}^\circ$) & 6.45803 & 0.0284 & 0.0538 & 0.0608 & 314.2 & 0.12 \\[1em]

    $t_C$(\unit{\us}) & 15.376 & 0.00245 & 0.00577 & 0.00627 & 171.9 & 0.0124 \\
    $H_C$(\unit{\mm}) & 45.4301 & 0.0219 & 0.0253 & 0.0334 & 28 & 0.0685 \\[1em]

    $t_A$(\unit{\us}) & 25.404 & 0.00126 & 0.00577 & 0.00591 & 1906.8 & 0.0116 \\
    $t_B$(\unit{\us}) & 51.5512 & 0.0015 & 0.00577 & 0.00596 & 1008.8 & 0.0117 \\
    $t_D$(\unit{\us}) & 33.0688 & 0.199 & 0.00577 & 0.199 & 4 & 0.553 \\[2mm]

    $x_A$(\unit{\mm}) & 28.9 & 0.0775 & 0.0577 & 0.0966 & 9.7 & 0.216 \\
    $x_B$(\unit{\mm}) & 86.94 & 0.0245 & 0.0577 & 0.0627 & 171.9 & 0.124 \\
    $x_D$(\unit{\mm}) & 107.02 & 0.0374 & 0.0577 & 0.0688 & 45.7 & 0.139 \\

    $c$(\unit{\mm\per\us}) & 3.2194 & 0.00412 & 0.00629 & 0.00752 & 136.8 & 0.0149 \\
    $t_0$(\unit{\us}) & 7.74638 & 0.0773 & 0.0985 & 0.125 & 32.1 & 0.255 \\
    $L_0^{'}$(\unit{\mm}) & 12.4348 & 0.133 & 0.217 & 0.254 & 202.6 & 0.501 \\
    $H_D$(\unit{\mm}) & 29.1118 & 0.233 & 0.0712 & 0.244 & 5.1 & 0.622 \\
    $L_D$(\unit{\mm}) & 90.9241 & 0.233 & 0.123 & 0.264 & 7.6 & 0.613 
\end{longtable}
\section{结果分析}
\subsection{数据分析}
\subsubsection{超声波频率和波长}
查阅讲义可知:超声波的频率大于 \SI{2d5}{\Hz},波长小于 \SI{20}{\mm}\par
\sisetup{
    uncertainty-mode=separate,
}%
根据实验数据计算得出的超声波频率为 \SI{2.685(31)}{\MHz},纵波波长为 \SI{2.322(27)}{\mm},横波波长为 \SI{1.175(14)}{\mm},与讲义和事实相符。
\subsubsection{探头延迟}
查阅资料可知:斜探头发射声波的角度不同,声波传播路径较长,相对于直探头延迟可能较高。\par
根据实验数据计算得出的直探头延迟为 \SI{0.798(43)}{\us},斜探头延迟为\SI{6.246(27)}{\us},与资料和事实相符。
\subsubsection{超声波波速}
查阅讲义可知:铝中纵波声速参考值为 \SI{6.27}{\mm\per\us},铝中横波声速参考值为 \SI{3.20}{\mm\per\us}。\par
本次实验数据计算得出的纵波声速为 \SI{6.233(13)}{\mm\per\us},与参考值相差不大,横波声速为 \SI{3.1552(29)}{\mm\per\us},与参考值相差不大。由此可见,本次实验操作严谨,测量准确。
\subsubsection{斜探头的前沿距离}
本次实验数据计算得出的前沿距离为 \SI{17.7(2)}{\mm},相对误差小,操作严谨。
\subsubsection{斜探头的折射角}
本次实验数据计算得出的折射角为 \SI{44.42(27)}{\degree},相对误差小,操作严谨。
\subsubsection{声束扩散角}
查阅资料得知:超声波探头的声束扩散角受多种因素影响,包括设计、频率、尺寸、材料以及超声波在介质中的传播速度等。对于焦深相对较小的超声波探头(如压电探头),其声束扩散角度可能在几度到十几度之间。这种探头可以产生较为集中的声束,有助于提高检测的分辨率和灵敏度。\par
本次实验数据计算得出的直探头的声束扩散角为 \SI{14.46(24)}{\degree},斜探头的声束扩散角为\linebreak \SI{6.46(12)}{\degree},处于资料指出的压电探头的声束扩散角度范围之内,符合事实。
\subsubsection{缺陷深度}
本次实验数据计算得出的缺陷 C 的深度为 \SI{45.430(69)}{\mm},而试样的缺陷 C 深度约为 \SI{45}{\mm},与计算结果十分接近。
\subsubsection{缺陷内超声声速和探头延迟}
在最后一个实验中计算得到的声速为 \SI{3.219(15)}{\mm\per\us},探头延迟为 \SI{7.75(26)}{\us},与之前实验测得的横波声速和探头延迟有一定差距,但声速接近参考值。两次实验所测得的数据存在差距的原因可能在于第二次实验是通过折射角间接计算声速和延迟的,测量量较少,导致误差较大。
\subsubsection{前沿距离、缺陷 D 的垂直深度和水平距离}
本次实验数据计算得出的探头的前沿距离为 \SI{12.43(51)}{\mm},与前面计算得出得结果相差大,可能是因为计算方法的不同,或者是定义不同导致的。缺陷 D 的垂直深度为 \SI{29.11(63)}{\mm},水平距离为 \SI{90.92(62)}{\mm},误差较小,操作严谨。
\subsection{误差分析}
\subsubsection{系统误差}
\begin{enumerate}
    \item 环境:实验当天的温度、湿度所带来的误差。查阅资料得知,温度会在一定程度下影响超声波的传播速度,进而影响实验测量。
    \item 仪器:测量中存在的误差来源包括刻度尺、示波器和探头。示波器的发射和接收器之间可能存在驻波场,但这种场并非严格的驻波场。这种情况可能导致信号的衰减和失真,因为部分超声波信号可能会在传播过程中被反射或散射,而不是完全被接收。这会导致接收到的信号强度降低,从而影响检测的灵敏度和可靠性。此外,示波器可能导致信号形状的改变、波形的扭曲或信号的重叠,进而影响信号的正确读取和分析
    \item 试样:试样中不同组成成分的差异会引起测量误差。查阅资料得知,试样中的杂质和结构不同时,超声波的声速也会发生变化。不同试样内部的杂质和缝隙等不致密因素的不同,会导致测量的超声声速与参考值存在一定差异。
\end{enumerate}
\subsubsection{偶然误差}
\begin{enumerate}
    \item 在用刻度尺测量各个长度时,可能会由于估读而产生读取数据的误差。
    \item 测量声束扩散角的实验中需要找到极大值和振幅减小一半的位置,可能会由于选取示波器上波形的位置而产生主观的读数误差。
\end{enumerate}
\subsubsection{减小误差的方法}
\begin{enumerate}
    \item 环境:可以通过测量当天的温度、湿度,并引入温度、湿度的修正系数来减小环境对声速所带来的误差。
    \item 仪器:可以使用测量精度更高的仪器来减小仪器所带来的测量误差。
\end{enumerate}
\section{思考题}
\begin{enumerate}
    \item \thinking{测量斜探头延迟和横波声速的时候,为什么斜探头打在圆弧面上,只有超声横波?}{%
    当超声波从一个介质进入另一个介质时,根据声阻抗匹配原理,超声波会发生折射和反射。斯涅尔定律指出,对于两个介质界面,入射角与折射角的正弦之比等于两个介质中声速的比值。从试样内反射回来的超声纵波入射角大于第一临界角,折射角大于 90 度,故纵波不会进入第二个介质,因此斜探头接收不到超声纵波。}
    \item \thinking{如果将待测试块从铝试块更换为钢试块,对同一斜探头测量到的延迟和入射点是否一样?为什么?}{%
    一样。延迟是探头本身的属性,与待测样品无关。入射点也是如此。测量时使用同一探头,延迟和入射点应该是一样的。}
\end{enumerate}