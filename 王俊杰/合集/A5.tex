\chapter{动态悬挂法测量金属材料杨氏模量}
\section{实验目的}
    \begin{enumerate}
        \item 理解动力学振动法测量材料杨氏模量的基本原理;
        \item 熟悉示波器的使用,学会用示波器观察信号和识别共振; 
        \item 学会用外延法处理实验数据,理解本实验采用外延法的原因。
    \end{enumerate}
\section{实验原理}%简单描述,含必要的公式和附图;
    材料加工制备或者工程应用中都会受到外力作用,受力后材料会发生变形:外力较小时发生弹性变形,持续增加外力逐渐发生塑性变形甚至断裂。在这些变形中,弹性变形首先发生,是其他变形的先行阶段,同时在塑性变形中也会伴随着弹性变形。对于理想的弹性变形,在弹性限度内材料所受应力与其应变保持线性函数关系,满足胡克定律,即弹性限度内应力、应变比值为一确定常量,定义该常数为材料的杨氏模量,表征材料抵抗弹性形变的能力。杨氏模量是材料最具特征的力学性能参数,是材料在实际工程设计和机械设计中极为重要的参考量。\par
    典型而常用的拉伸法即为静态法测量材料的杨氏模量,但这种静态拉伸法仅适用于材料形变量大,延展性好的情况,对于脆性材料如玻璃、陶瓷等不适用。动态法适用材料范围广,可测量脆性材料,适用于不同的温度环境,测量结果稳定,理论同实验吻合度高。这些测量上的优越性使得动态法测量杨氏模量在实际应用中应用非常广泛,是国家标准指定的一种测量杨氏模量的方法。\par
    本实验即采用动态法测量不同金属试样的杨氏模量,具体方法为:将一根截面均匀的棒状试样通过悬线悬挂在两只传感器(一只激振,一只拾振)下面,试样端部不受其他外力,满足自由振动,由此对试样进行激振从而检测出其振动时的固有基频,而后可测得材料的杨氏模量。\par
    在一定条件下,对于具有确定材料组成和确定形状的待测试样,其固有频率的大小既与待测试样自身的几何形状、尺寸和质量有关,还与组成试样的材料本身的杨氏模量直接相关。因此,如果我们通过实验测得某试样在一定温度下的固有频率,就可以通过简单的几何尺寸和质量的测量算得组成试样的材料在此温度下的杨氏模量。由此,基于此方法对材料杨氏模量的测量关键问题在于如何准确得测量试样本身的固有频率。\par
    对试样棒建模可得到原始方程:
    \begin{equation}
        \frac{\partial^4y}{\partial x^4}=\frac{\rho S}{EJ}\cdot\frac{\partial^2y}{\partial t^2}
    \end{equation}
    式中 $\rho,S,E,J$ 分别表示试样的材料密度、试样棒的截面积、试样材料的杨氏模量和试样棒某一截面的惯量矩($J=\int y^2ds$)。\par
    最后我们得到,圆柱形棒材的杨氏模量 E 可以表示为:
    \begin{equation}
        E=1.6067\frac{l^3m}{d^4}f^2\label{eq:A5.2}
    \end{equation}
    式中 $l$ 为棒长、 $d$ 为圆形棒的截面直径,单位 \unit{\m} ;$m$ 为棒的质量,单位 \unit{\kg} ; $f$为试样固有频率,单位 \unit{\Hz}。 所以,如果实验中测定了试样在不同温度时的固有频率 $f$,即可通过上式计算出试样在对应温度下的杨氏模量 $E$。在 SI 中杨氏模量的单位为\unit{\N\per\m^2}。\par
    微分方程的解还说明了:试样作基频振动时,存在距离端面分别为 $0.224l$ 和$0.776l$的两个节点。显然,节点处是不振动的,阻尼为零,此时无阻尼自由振动的振动频率就是试样的固有频率,即当支撑点为节点时,测得的共振频率就是试样的固有频率。但是,因为节点处不振动无法实现激振,故而实验时悬丝不能吊挂在试样节点上而只能挂在节点附近。在实际测量中,悬丝和悬挂点均会对试样的自由振动产生阻尼,因而,所检测到的共振频率会随着悬挂点位置的不同而变化,且悬挂点偏离节点越远,可检测的信号越强,但共振频率将偏离固有频率越大。所以,要测量振动试样的固有频率(基频频率),需要通过适当的数据处理获得试棒悬挂点为节点处时,试棒做无阻尼自由振动的基频频率。 
\section{实验仪器}%规格及参数
    DY-A型金属动态杨氏模量测定仪、金属动态杨氏模量测试台、待测金属试样、游标卡尺、螺旋测微计、天平、示波器等。 \par
    动态杨氏模量测量仪作为信号发生器输出等幅正弦波电信号,信号经放大器放大,通过悬线上方的激振传感器将电信号转变为机械振动,由悬线把机械振动传递给待测试样,待测试样因受到悬线作用的横向作用力而产生横振动。试样的横振动通过另一端的悬线传递给拾振传感器,通过拾振传感器将试样横振动转变为周期电信号,经放大器放大后输入示波器。当信号发生器的输出频率不等于此悬挂状态下试样的固有频率时,试样不发生共振,示波器上几乎没有信号波形显示或波形很小;当信号发生器的输出频率几乎接近试样相应悬挂状态下的固有频率时,试样发生共振,这时,示波器上波形会突然增大,此时信号发生器的输出信号频率,即为此时的共振频率,也是试样在此悬挂点悬挂时的固有频率。
\section{实验过程}%简述主要过程和实验内容
    \begin{enumerate}
        \item 首先测量出待测试样的长度 $l$,直径 $d$ 和质量 $m$ ,测量用具分别为游标卡尺、螺旋测微器和电子天平;
        \item 根据 \eqref{eq:A5.2} 估算待测试样的固有频率。已知室温下待测试样不锈钢棒材和铜棒材的杨氏模量标准值分别为 \SI{2E+11}{\N\per\m^2} 和 \SI{1.2E+11}{\N\per\m^2},先估算试样的固有频率f,以便寻找共振点。
        \item 要使试样共振频率为待测试样的固有频率,试样本身须作无阻尼自由振动,要求在操作中,悬挂点须在两个节点位置(距离端面分别为 $0.224l$ 和 $0.776l$ 处)。但实际上,悬挂点在节点处时因节点处 $y=0$ 无法实现试样激发振动。因此,在实际操作中,实际的吊扎位置要偏离节点;
        \item 在偏离节点且距离试样两端面等距离处选择两位置,记录两点位置x并作为悬挂点在此两位置处悬挂待测试棒。室温下测量该悬挂位置待测金属棒的共振频率f;试样共振状态的建立需要有一个过程,且共振峰十分尖锐。因此共振点附近调节信号频率时,必须十分缓慢的进行
        \item 在节点的两侧分别选择不同位置对待测试棒进行悬挂,之后按照4相同的方法,测出待测金属棒在不同悬挂点悬挂时的共振频率。要求节点两侧待测量悬挂位分别两个及以上;
        \item 以悬挂点$\frac{x}{l}$为横坐标,上述4和5测得的共振频率 $f$ 为纵坐标做图,可得悬挂点位置和共振频率之间的变化曲线。根据曲线变化规律,采用内插法或外延法处理实验数据,当 $x$ 逼近 $0.224l$ 时即可得到悬挂点在节点处的共振频率,根据前述基本原理,此时的共振频率为试样无阻尼自由振动的基频,亦即试样的固有频率;代入 \eqref{eq:A5.2} 求得试样杨氏模量,与标准比对分析和讨论实验误差。 
    \end{enumerate}

\section{实验数据}
\subsection{不锈钢棒}
根据\seqref{eq:A5.2} 可估算出不锈钢棒的共振频率为 \SI{1156.25}{\Hz},节点位置为 \SI{33.63}{\mm}
\begin{table}[!ht]
    \caption{不锈钢棒几何尺寸测量数据}\label{tab:A5.steelSize}
    \begin{tabular}{c *{5}{S[table-format=3.3]} S[table-format=3.4]} \toprule
        次数 & {1} & {2} & {3} & {4} & {5} & {平均值} \\ \midrule
        长度(\unit{\mm}) & 150.12 & 150.10 & 150.16 & 150.14 & 150.10 & 150.124 \\ 
        直径(\unit{\mm}) & 5.92 & 5.90 & 5.90 & 5.88 & 5.88 & 5.896 \\ 
        质量(\unit{\g}) & 33.349 & 33.350 & 33.350 & 33.350 & 33.350 & 33.3498 \\ \bottomrule
    \end{tabular}
\end{table}

\begin{table}[!ht]
    \caption{不锈钢棒共振频率测量数据}\label{tab:A5.steelFreq}
    \begin{tabular}{c *{5}{S[table-format=4.1]}} \toprule
        次数 & {1} & {2} & {3} & {4} & {5} \\ \midrule
        $x$(\unit{\mm}) & 10.0 & 20.0 & 25.0 & 45.0 & 55.0 \\ 
        $f$(\unit{\Hz}) & 1178.9 & 1175.0 & 1172.6 & 1176.7 & 1180.1 \\\bottomrule
    \end{tabular}
\end{table}
根据以上数据,在 OriginPro 软件中作图并拟合可得
\begin{figure}[!ht]
    \caption{不锈钢棒共振频率拟合曲线}
    \includegraphics[width=0.8\textwidth]{img/A5/steelFreq.pdf}
\end{figure}

\subsection{铜棒}
根据\seqref{eq:A5.2} 可估算出不锈钢棒的共振频率为 \SI{1156.25}{\Hz},节点位置为 \SI{33.63}{\mm}
\begin{table}[!ht]
    \caption{铜棒几何尺寸测量数据}\label{tab:A5.copperSize}
    \begin{tabular}{c *{5}{S[table-format=3.3]} S[table-format=3.4]} \toprule
        次数 & {1} & {2} & {3} & {4} & {5} & {平均值} \\ \midrule
        长度(\unit{\mm}) & 149.90 & 149.88 & 149.90 & 149.88 & 149.88 & 149.888 \\ 
        直径(\unit{\mm}) & 5.88 & 5.86 & 5.88 & 5.88 & 5.86 & 5.872 \\ 
        质量(\unit{\g}) & 34.980 & 34.980 & 34.979 & 34.980 & 34.979 & 34.9796 \\ \bottomrule
    \end{tabular}
\end{table}

\begin{table}[!ht]
    \caption{铜棒共振频率测量数据}\label{tab:A5.copperFreq}
    \begin{tabular}{c *{5}{S[table-format=4.1]}} \toprule
        次数 & {1} & {2} & {3} & {4} & {5} \\ \midrule
        $x$(\unit{\mm}) & 10.0 & 20.0 & 25.0 & 45.0 & 55.0 \\ 
        $f$(\unit{\Hz}) & 839.9 & 836.4 & 834.1 & 840.3 & 844.2 \\ \bottomrule
    \end{tabular}
\end{table}
根据以上数据,在 OriginPro 软件中作图并拟合可得
\begin{figure}[!ht]
    \caption{铜棒共振频率拟合曲线}
    \includegraphics[width=0.8\textwidth]{img/A5/copperFreq.pdf}
\end{figure}
\section{结果分析}
根据\seqref{eq:A5.2} 可计算
\begin{align*}
    E_\text{不锈钢棒} &= 1.6067\times \frac{150.124^3 \times 33.3498}{5.896^4}\times 1171.34^2= \SI{1.937d11}{\N \per \square \m} \\
    E_\text{铜棒} &= 1.6067\times \frac{149.888^3 \times 34.9796}{5.872^4}\times 833.153^2= \SI{1.105d11}{\N \per \square \m}
\end{align*}
相对误差为
\begin{align*}
    \Delta E_\text{不锈钢棒} &= 3.2\% \\
    \Delta E_\text{铜棒} &= 7.9\%
\end{align*}
\subsection{误差分析}
\subsubsection{系统误差}
\begin{enumerate}
    \item 在数据分析中我们可以得到曲线最小值点分别为 $x=31.2368$ 和 $x=30.1289$,这两个点与理论的 $x \approx 33.6$ 具有一定的差异。是因为拟合所采取的方程式四次曲线,但实际的曲线可能并非如此,甚至可能偏差较大,造成拟合结果看似十分完美。
    \item 实验当天的温度、湿度所带来的误差。
    \item 仪器本身精度的影响,存在一定的误差。
    \item 不锈钢棒和铜棒的加工工艺可能会对杨氏模量产生一定的影响,使不同的不锈钢棒和铜棒具有不完全相同的杨氏模量。
    \item 本实验计算的杨氏模量所需要的固有频率是由拟合曲线出来的,与事实存在偏差。
\end{enumerate}
\subsubsection{偶然误差}
\begin{enumerate}
    \item 在读取频率时,可能会由于主观取舍固有频率的波动值而产生一定的误差。在悬挂时,也难以完全准确地固定悬挂点在指定位置。
    \item 在对数据进行四舍五入和拟合分析时,可能会产生偶然误差。
\end{enumerate}

\section{思考题}
\begin{enumerate}
    \item \thinking{什么是杨氏模量?杨氏模量的意义?}{杨氏模量是材料力学中的一个重要参数指在材料受到拉伸或压缩时,单位面积内的应力与相应的应变之间的比值,通常用符号 $E$ 表示。换言之,杨氏模量衡量了材料在受力时的刚度,即材料对外部应力的响应程度。杨氏模量的意义在于:
    \begin{enumerate}
        \item 比较不同材料的应力应变性质:通过比较不同材料的杨氏模量,可以评估它们在特定应力条件下的相对性能。这对于工程设计和材料选择至关重要,因为杨氏模量直接影响到材料在使用过程中的变形和稳定性。
        \item 设计应用的指导:杨氏模量是工程设计中的重要参数之一。它用于计算结构在受力时的变形、应力分布等,有助于工程师设计出更加稳定和符合要求的结构。
    \end{enumerate}
    }
    \item \thinking{试讨论,试样的长度 $l$、直径 $d$、质量 $m$、共振频率 $f$ 分别应该采用什么规格的仪器测量?为什么?}{
    }\begin{enumerate}
        \item 试样的长度 $l$ 和直径 $d$ 用最小刻度 \SI{0.02}{\mm} 的游标卡尺测量,这样可以使测量尽可能准确,减少实验误差。
        \item 质量 $m$ 用最小刻度 \SI{0.001}{\g} 的电子天平测量,这样可以使测量尽可能准确,减少实验误差。
        \item 共振频率 $f$ 用带宽大于 \SI{10}{\kHz} 的示波器测量,这样测量的频率满足五倍法则,能更加准确的找到相应的共振峰。如果精度不够的话,可能会出现无论怎么变化都得不到共振峰的情况。
    \end{enumerate}
\end{enumerate}

\chapter{基于迈克尔逊干涉测量金属材料线膨胀系数}
\section{实验目的}
    \begin{enumerate}
        \item 观察材料的线膨胀现象,了解测量材料线膨胀系数的基本原理;
        \item 了解光学干涉现象及迈克尔逊干涉的基本原理及调节方法; 
        \item 理解测量微小尺寸变化的方法。
    \end{enumerate}
\section{实验原理}%简单描述,含必要的公式和附图;
    由于微观粒子热运动的存在,绝大多数材料宏观上存在“热胀冷缩”现象,即物体存在受热时会膨胀,受冷时会收缩的特性。这种随温度变化而产生宏观尺寸变化的特性在材料的实际应用中应予以充分考虑,否则将产生不利影响,如因“热胀冷缩”可能会引起工程结构的损坏、仪表失灵等。线膨胀系数为了表征物体随温度变化其长度变化程度而引入的物理量,可定量分析材料热膨胀问题,是衡量材料热稳定性的重要指标。目前,对微小长度和其变化量的测量方法按照基本原理可分为三类:利用杠杆原理、利用光学干涉原理和利用螺旋测微原理直接测量。在几种测量方法中,利用光学干涉现象测量微小尺寸的结果精度最高,本实验即基于迈克尔逊干涉现象测量金属材料随温度变化在确定方向的伸长量,从而可得其线膨胀系数。\par
    线膨胀系数在数值上定义为固体材料每升高 \SI{1}{\degreeCelsius} 时单位长度的伸长量。对于同一材料,在不同的温度区间,线膨胀系数是不同的,但在温度变化不太大的情况下,通常认为材料的线膨胀系数可近似为常量。假设在确定的温度范围内,一固态物体在确定方向上温度为 $t-0$(单位:\unit{\degreeCelsius})时长度为 $L_0$,对物体进行升温,当温度为 $T$ 时,其长度为 $L$,伸长量 $\varDelta L=L-L0$。实验表明,物体在确定方向上的单位伸长量与温度增量$\varDelta T$近似成正比,即:
    \begin{equation}
        \frac{\varDelta L}{L_0} = \alpha \times \varDelta T \Longrightarrow  \alpha=\frac{\varDelta L}{L_0 \varDelta T}
    \end{equation}
    $\alpha$ 即为该物体在此温度区间内的线膨胀系数,表征了组成材料在此确定方向上的受热膨胀(或收缩)的程度。温度变化不大时,材料的线膨胀系数 $\alpha$ 可近似为常数。\par
    本实验基于迈克尔逊干涉实验测量线膨胀系数,具体方法为:将待测金属试样置于左下侧温控炉内并通过温控炉在一定温度范围内改变待测试样温度,试样将受热发生膨胀而伸长,推动反射镜向上移动,从而使迈克尔逊干涉中其中的一路的光程发生变化,因而,两路光线的光程差改变,由此可实现干涉条纹的动态观察。\par
    经推导得到本次实验计算用公式:
    \begin{equation}
        \alpha=\frac{N\frac\lambda2}{L_1(t_2-t_1)}\label{eq:A5.alpha}
    \end{equation}
\section{实验仪器}%规格及参数
    热膨胀实验仪
\section{实验过程}%简述主要过程和实验内容
    \begin{enumerate}
        \item 准备待测试样,将待测试样放入温控炉中;
        \item 根据迈克尔逊干涉光路原理调节光路,观察干涉条纹
        \item 测量和读取数据
        \item 根据所测数据,计算待测试样的线膨胀系数
        \item 本次实验要求测量黄铜和硬铝两个试样并对测量数据做表记录         
    \end{enumerate}
\section{实验数据}
\subsection{原始数据}
\subsubsection{长度}
\begin{table}[!ht]
    \caption{长度测量原始数据}
    \begin{tabular}{*{7}{c}} \toprule
        测量次数 & 1 & 2 & 3 & 4 & 5 & 平均值  \\ \midrule
        黄铜(\unit{\mm}) & 149.90 & 149.92 & 149.90 & 149.88 & 149.92 & 149.904 \\ 
        硬铝(\unit{\mm}) & 149.86 & 149.84 & 149.80 & 149.82 & 149.82 & 149.828 \\ \bottomrule
    \end{tabular}
\end{table}
\subsubsection{干涉环变化}
\begin{table}[!ht]
    \caption{干涉环变化原始数据}
    \begin{tabular}{*{7}{c}}\toprule
        & \multicolumn{3}{c}{黄铜} & \multicolumn{3}{c}{硬铝} \\ 
        & $t$(\unit{\degreeCelsius}) & $\Delta t$(\unit{\degreeCelsius}) & $N$ & $t$(\unit{\degreeCelsius}) & $\Delta t$(\unit{\degreeCelsius}) & $N$  \\ \midrule
       1 & 23.4 &     &    & 27.0 &     &    \\ 
       2 & 24.5 & 1.1 & 10 & 27.9 & 0.9 & 10 \\ 
       3 & 25.5 & 1.0 & 10 & 28.9 & 1.0 & 10 \\ 
       4 & 26.6 & 1.1 & 10 & 29.8 & 0.9 & 10 \\ 
       5 & 27.8 & 1.2 & 10 & 30.8 & 1.0 & 10 \\ 
       6 & 28.8 & 1.0 & 10 & 31.7 & 0.9 & 10 \\ \bottomrule
    \end{tabular}
\end{table}
\subsection{数据处理}
由于直接对 $\Delta t$ 取平均值会丢失数据,故使用线性拟合处理实验数据,以试样温度 $t$ 为纵坐标,干涉环总变化数 $N_\text{总}$ 为横坐标,根据\seqref{eq:A5.alpha} 可得
\begin{equation}
    \alpha = \frac{\lambda}{2 L_1 k} \label{eq:A5.5}
\end{equation}
其中 $k$ 为拟合直线的斜率 \par
将数据放入 OriginPro 中处理可得
\begin{figure}[!ht]
    \begin{floatrow}
        \ffigbox[0.49\textwidth]{\caption{黄铜数据拟合结果}}{\includegraphics[width=0.49\textwidth]{img/A5/Copper.pdf}}
        \ffigbox[0.49\textwidth]{\caption{硬铝数据拟合结果}}{\includegraphics[width=0.49\textwidth]{img/A5/Aluminum.pdf}}
    \end{floatrow}
\end{figure}
\begin{align}
    k_\text{黄铜}&=\SI{0.1086(24)}{\degreeCelsius}\\
    k_\text{硬铝}&=\SI{0.0946(24)}{\degreeCelsius}
\end{align}
带入\seqref{eq:A5.5} 可得
\begin{align}
    \alpha_\text{黄铜}&=\SI{1.944(43)d-5}{\per\degreeCelsius}\\
    \alpha_\text{硬铝}&=\SI{2.233(57)d-5}{\per\degreeCelsius}
\end{align}
不确定度汇总得到表 \ref{tab:A5.uncertainty2}
\begin{table}[!ht]
    \caption{不确定度}\label{tab:A5.uncertainty2}
    \begin{tabular}{*{4}{c}} \toprule
        & $k$(\unit{\degreeCelsius}) & $L$(\unit{\mm}) & $\alpha$(\unit{\per\degreeCelsius}) \\ \midrule
        黄铜 & 0.0024 & 0.0075 & \num{5.7d-7} \\ 
        硬铝(\unit{\mm}) & 0.0024 & 0.0102 & \num{4.3d-7} \\ \bottomrule
    \end{tabular}
\end{table}
\section{结果分析}
\subsection{实验数据}
查阅讲义得知,铜棒的线膨胀系数参考值为 \SI{2.08d-5}{\per\degreeCelsius},铝棒的线膨胀系数参考值为 \SI{2.36d-5}{\per\degreeCelsius}。可以看出,相对误差的值在合理范围之内而且相对较小,体现了实验操作的准确严谨。实验标准误差也很小,可以看出实验的精密性非常好。
\subsection{误差分析}
\subsubsection{系统误差}
\begin{enumerate}
    \item 实验当天的温度、湿度所带来的误差。
    \item 仪器本身精度的影响,如温控仪本身只能取到小数点后一位,会产生一定的误差和较大的不确定度。
    \item 不锈钢棒和铜棒的成分和组分不均可能会对实验造成一定影响。
    \item 实验原理的线膨胀系数使用高阶多项式表示,但计算的只是常数项,忽略高阶项计算得到常数项的可能不准确。
\end{enumerate}
\subsubsection{偶然误差}
\begin{enumerate}
    \item 在读取温度时,由于较难界定何时达到与初始温度相同的干涉环状态,因此,何时读取最终的温度具有一定的主观性,进而产生偶然误差。
    \item 在对数据进行四舍五入计算时,可能会产生偶然误差。
\end{enumerate}
\section{思考题}
\begin{enumerate}
    \item \thinking{测量材料线膨胀系数的意义并举例说明。}{线膨胀系数是描述材料在温度变化下尺寸变化的参数,通常用于工程设计和材料选择中。以下是材料线膨胀系数在不同方面的意义:
    \begin{enumerate}
        \item 工程结构稳定性:在许多工程应用中,例如建筑、桥梁、管道系统等,材料的尺寸稳定性对结构的安全和稳定至关重要。温度变化会引起材料的线膨胀或收缩,如果这些变化没有被适当考虑,可能会导致结构的不稳定或者产生过度应力,最终影响结构的性能和寿命。
        \item 材料选择和设计考虑:在工程设计过程中,了解材料的线膨胀系数可以帮助工程师选择合适的材料,并确保设计的结构在不同温度条件下保持稳定。在高温环境下工作的设备或构件,需要选择线膨胀系数较低的材料,以减少因温度变化引起的尺寸变化对结构的影响。除此之外,比如温度在夏季和冬季会发生较大变化,在构建桥梁时就需要考虑材料的线膨胀系数,以及长度可能会随着温度的变化而发生微小的变化。如果不考虑这种长度变化,可能会导致桥梁在不同温度下出现过度应力或结构不稳定的问题。
        \item 工程测量和校准:许多测量设备和仪器在使用时需要考虑温度对其尺寸和性能的影响。了解材料的线膨胀系数可以帮助校准这些设备,以确保它们在不同温度下提供准确的测量结果。
        \item 材料研究和开发:在材料科学和工程领域,研究材料的线膨胀系数可以帮助开发新的材料,以满足特定工程应用的需求。通过调整材料的成分和结构,可以改变其线膨胀系数,从而使其更适用于特定的温度环境下。
    \end{enumerate}}
    \item \thinking{分析实验中影响实验结果的因素有哪些。}{
    \begin{enumerate}
        \item 光源稳定性:光源的频率或强度的微小变化可能导致测量误差,所以实验中使用的光源必须具有稳定的频率和强度,以确保干涉图样的稳定性和准确性。
        \item 环境温度:金属材料的线膨胀系数与温度密切相关,因此必须确保实验室环境的温度稳定。即使微小的温度变化也可能影响金属材料的尺寸,进而影响干涉图样的解释和测量结果。
        \item 光路稳定性:迈克尔逊干涉仪的光路必须保持稳定,以确保干涉条纹的清晰度和稳定性。任何光路中的微小振动或调整都可能导致干涉图样的变化。
    \end{enumerate}}
\end{enumerate}