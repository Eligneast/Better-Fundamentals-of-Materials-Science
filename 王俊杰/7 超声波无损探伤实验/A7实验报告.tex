\documentclass[a4paper,utf8]{article}
\usepackage{graphicx}
\usepackage[heading,fancyhdr]{ctex}
\usepackage{amsmath,amssymb,geometry,ulem}
\usepackage{array,tabularx,tabulary,mhchem,xspace}
\usepackage{floatrow,subfig,multirow,bigstrut}
\usepackage{siunitx,booktabs,longtable,nameref}
\lineskiplimit=1pt
\lineskip=3pt
\geometry{
    top=25.4mm, 
    left=25mm, 
    right=25mm, 
    bottom=25mm,
    headsep=5.9mm,
}
\ctexset{
    chapter = {
        name = {实验,},
        beforeskip = {-23pt}
    }
}
\newcommand{\fgref}[1]{图~\ref{#1}\xspace}
\newcommand{\seqref}[1]{式~(\ref{#1})}
\newcommand{\expinfo}[6][无]{
    {\zihao{-3}\bfseries\songti
    实验名称:\uline{\hfill\mbox{#2}\hfill} \\[2.9mm]
    学\quad 号:\uline{\makebox[25mm]{#3}}\hfill
    姓\quad 名:\uline{\makebox[25mm]{#4}}\hfill
    班\quad 级:\uline{\makebox[25mm]{#5}} \\[2.9mm]
    合作者:\uline{\makebox[25mm]{#1}} \hfill
    桌\quad 号:\uline{\makebox[25mm]{}}\hfill\makebox[25mm+4em]{}\\[2.9mm]
    指导教师:\uline{\makebox[30mm]{#6}}\hfill\mbox{} \\[2.9mm]
    实验日期:\uline{\makebox[30mm]{}}\hfill\mbox{} \\[58.7mm]
    }
}%\expinfo[合作者]{实验名称}{学号}{姓名}{班级}{指导教师}
\newcommand{\pointingbox}{
    {\zihao{4}\bfseries\songti%
    实验考核\\[3mm]
    \extrarowheight=3mm
    \begin{tabularx}{150mm}{|X|X|X|X|X|}\hline
        \hfil 项目 \hfil  & \hfil 实验预习 \hfil & \hfil 实验过程 \hfil & \hfil 分析与讨论 \hfil & \hfil 总评 \hfil \\[3mm] \hline
        \hfil 评价 \hfil &  &  &  &  \\[3mm] \hline
    \end{tabularx}
    }
}
\newcommand{\derivative}[2]{\frac{\mathrm{d} #1}{\mathrm{d} #2}}
\newcommand{\thinking}[2]{\textbf{#1}\\
答:\begin{minipage}[t]{0.85\textwidth}
    #2
\end{minipage}}

\pagestyle{fancy}
\fancyhf{}
%\fancyhead[C]{材料科学基础实验}
%\fancyfoot[C]{\thepage}
\fancyhead[EC]{\leftmark} \fancyhead[OC]{\rightmark}
\fancyhead[EL,OR]{\thepage}
\fancypagestyle{plain}{\renewcommand{\headrulewidth}{0pt}\fancyhf{}}

\newcounter{Rownumber}
\newcommand*{\Rown}{\stepcounter{Rownumber}\theRownumber}
\newcounter{sample}
\newcommand*{\Sam}{\stepcounter{sample}\thesample}
\newcounter{Fignumber}
\newcommand*{\Fign}{\stepcounter{Fignumber}\theFignumber}

\newcommand*{\resetRown}{\setcounter{Rownumber}{0}}
\newcommand{\qrange}[3]{\qtyrange[range-phrase = \text{$\sim$},range-units =single]{#1}{#2}{#3}}
\floatsetup[table]{capposition=top}
\newcolumntype{C}{>{\hfil}X<{\hfil}}
\renewcommand{\Nameref}[1]{\textbf{\ref{#1}~\nameref{#1}}}
\newcommand{\TTR}[0]{\watt\per\m\per\K} %导入导言
\begin{document}
\begin{center}
    {\mbox{}\\[7em]\zihao{2}\bfseries\songti%
    材料科学基础实验报告}\\[34mm]
    \expinfo{超声波无损探伤实验}{22301056}{王俊杰}{22 材物}{李继玲}
    \pointingbox
\end{center}\newpage
\section{实验目的}
    \begin{itemize}
        \item 学习超声波的产生原理及特点,了解超声波探伤仪的工作原理及使用方法 
        \item 学习超声探头指向性原理及其实际应用
        \item 认识超声换能器及超声波探头的不同种类
        \item 了解超声波的传播、波型和波型转换原理及超声波声速的测量方法
    \end{itemize}
\section{实验原理}%简单描述,含必要的公式和附图;
    超声波是频率高于 20kHz 的机械波,具有穿透力强、传播方向性好等特点,实验所用探头通过压电晶片的逆压电效应,将电能转为机械能以产生脉冲超声波。超声波探头结构复杂,包括压电晶片、保护膜、匹配电感等部分,根据不同结构和应用情况可分为直探头、斜探头和可变角探头。超声波的指向性是指探头发射的声束扩散角大小,与波长、频率及探头内压电晶片尺寸有关。声束扩散角越小,指向性越好,测量精度越高。指向性大小可用公式表示:
    \begin{equation}
        \theta=2\sin^{-1}\left(1.22\frac\lambda D\right)
    \end{equation}\par
    超声波根据波形可分为超声纵波、超声横波和超声表面波三种,在介质界面发生反射、折射和波形转换时需满足斯特令定律:
    \begin{equation}
        \begin{aligned}
            \text{反射:}\frac{\sin\alpha}C&=\frac{\sin\alpha_L}{C_{1L}}=\frac{\sin\alpha_S}{C_{1S}}\\
            \text{折射:}\frac{\sin\alpha}C&=\frac{\sin\beta_L}{C_{2L}}=\frac{\sin\beta_S}{C_{2S}}
        \end{aligned}
    \end{equation}\par
    测量超声波声速可采用直接或相对测量法。直接测量法利用超声波探头内部延迟时间和探头测量的人工反射体回波时间计算声速;相对测量法则通过测量两次反射回波的时间差来计算声速。
\section{实验仪器}%规格及参数
    JDUT-2 型超声波试验仪、DS1102E 双通示波器(100MHz)、直探头、斜探头、CSK-IB 试块、耦合剂(水)等。
\section{实验过程}%简述主要过程和实验内容
    \begin{enumerate}
        \item 组装实验仪器;
        \item 利用直探头测量脉冲超声纵波频率和波长
        \item 测量直探头延迟和间接测量法测量试块纵波声速
        \item 测量斜探头延迟、入射点和间接测量法测试块横波声速
        \item 测量斜探头的折射角
        \item 测量直探头和斜探头的声束扩散角
        \item 使用直探头探测缺陷深度
        \item 使用斜探头探测待测试块内部缺陷位置
    \end{enumerate}
\section{实验数据与分析}
\subsection{实验数据\label{sss:data}}
\begin{table}[!ht]
    \caption{CSK-IB 铝试块参数测量}\label{tab:length}
    \begin{tabularx}{\textwidth}{*{7}{C}} \toprule
        测量量 & 1 & 2 & 3 & 4 & 5 & 平均值 \\ \midrule
        $L_A$(\unit{\mm}) & 19.8 & 19.9 & 19.8 & 19.8 & 19.9 & 19.84 \\ 
        $L_B$(\unit{\mm}) & 50.0 & 50.0 & 50.1 & 50.0 & 50.0 & 50.02 \\ 
        $L_{AB}$(\unit{\mm}) & 30.2 & 30.1 & 30.3 & 30.2 & 30.1 & 29.86 \\[2mm]
        $H_A$(\unit{\mm}) & 20.3 & 20.2 & 20.4 & 20.3 & 20.3 & 20.30 \\
        $H_B$(\unit{\mm}) & 50.4 & 50.4 & 50.3 & 50.3 & 50.4 & 50.36 \\
        $H_{AB}$(\unit{\mm}) & 30.1 & 30.2 & 29.9 & 30.0 & 30.1 & 30.06 \\ \bottomrule
    \end{tabularx}
\end{table}

\begin{table}[!ht]
    \caption{利用直探头测量脉冲超声纵波频率和波长}\label{tab:exp1}
    \begin{tabularx}{\textwidth}{*{7}{C}} \toprule
        测量量 & 1 & 2 & 3 & 4 & 5 & 平均值 \\ \midrule
        $t$(\unit{\us}) & 1.50 & 1.47 & 1.49 & 1.50 & 1.49 & 1.49 \\ \bottomrule
    \end{tabularx}
\end{table}

\begin{table}[!ht]
    \caption{测量直探头延迟和间接测量法测量试块纵波声速}\label{tab:exp2}
    \begin{tabularx}{\textwidth}{*{7}{C}} \toprule
        测量量 & 1 & 2 & 3 & 4 & 5 & 平均值 \\ \midrule
        $t_1$(\unit{\us}) & 15.26 & 15.24 & 15.23 & 15.22 & 15.24 & 15.238 \\ 
        $t_2$(\unit{\us}) & 29.71 & 29.68 & 29.67 & 29.66 & 29.67 & 29.678 \\ \bottomrule
    \end{tabularx}
\end{table}

\begin{table}[!ht]
    \caption{测量斜探头延迟、入射点和间接测量法测试块横波声速}\label{tab:exp3}
    \begin{tabularx}{\textwidth}{*{7}{C}} \toprule
        测量量 & 1 & 2 & 3 & 4 & 5 & 平均值 \\ \midrule
        $t_3$(\unit{\us}) & 25.26 & 25.27 & 25.26 & 25.26 & 25.26 & 25.262 \\ 
        $t_4$(\unit{\us}) & 44.28 & 44.28 & 44.28 & 44.27 & 44.28 & 44.278 \\ \bottomrule
    \end{tabularx}
\end{table}

\begin{table}[!ht]
    \caption{测量斜探头入射点}\label{tab:exp4}
    \begin{tabularx}{\textwidth}{*{7}{C}} \toprule
        测量量 & 1 & 2 & 3 & 4 & 5 & 平均值 \\ \midrule
        $L$(\unit{\mm}) & 42.5 & 42.3 & 42.2 & 42.3 & 42.2 & 42.3 \\ \bottomrule
    \end{tabularx}
\end{table}

\begin{table}[!ht]
    \caption{测量斜探头的折射角}\label{tab:exp5}
    \begin{tabularx}{\textwidth}{*{7}{C}} \toprule
        测量量 & 1 & 2 & 3 & 4 & 5 & 平均值 \\ \midrule
        $L_{A1}$(\unit{\mm}) & 27.4 & 27.5 & 27.5 & 27.5 & 27.5 & 27.48 \\ 
        $L_{B1}$(\unit{\mm}) & 87.2 & 87.1 & 87.1 & 87.0 & 87.2 & 87.12 \\ \bottomrule
    \end{tabularx}
\end{table}

\begin{table}[!ht]
    \caption{测量直探头和斜探头的声束扩散角}\label{tab:exp6}
    \begin{tabularx}{\textwidth}{*{7}{C}} \toprule
        测量量 & 1 & 2 & 3 & 4 & 5 & 平均值 \\ \midrule
        $x_1$(\unit{\mm}) & 44.2 & 44.1 & 44.4 & 44.3 & 44.2 & 44.24 \\ 
        $x_2$(\unit{\mm}) & 57.1 & 57.0 & 57.0 & 56.9 & 57.1 & 57.02 \\ 
        $x_3$(\unit{\mm}) & 82.9 & 83.0 & 83.0 & 82.9 & 83.0 & 82.96 \\ 
        $x_4$(\unit{\mm}) & 94.1 & 94.2 & 94.1 & 94.1 & 94.0 & 94.10 \\ \bottomrule
    \end{tabularx}
\end{table}

\begin{table}[!ht]
    \caption{使用直探头探测缺陷深度}\label{tab:exp7}
    \begin{tabularx}{\textwidth}{*{7}{C}} \toprule
        测量量 & 1 & 2 & 3 & 4 & 5 & 平均值 \\ \midrule
        $t_C$(\unit{\us}) & 15.38 & 15.37 & 15.38 & 15.37 & 15.38 & 15.376 \\ \bottomrule
    \end{tabularx}
\end{table}

\begin{table}[!ht]
    \caption{使用斜探头探测待测试块内部缺陷位置}\label{tab:exp8}
    \begin{tabularx}{\textwidth}{*{7}{C}} \toprule
        ~ & 1 & 2 & 3 & 4 & 5 & 平均值 \\ \midrule
        $t_A$(\unit{\us}) & 25.404 & 25.400 & 25.404 & 25.404 & 25.408 & 25.404~~{} \\
        $t_B$(\unit{\us}) & 51.548 & 51.552 & 51.552 & 51.548 & 51.556 & 51.5512 \\
        $t_D$(\unit{\us}) & 33.264 & 33.272 & 33.272 & 33.264 & 22.272 & 31.0688 \\[2mm]
        $x_A$(\unit{\mm}) & ~~28.6 & ~~29.0 & ~~29.0 & ~~28.9 & ~~29.0 & ~~28.90 \\
        $x_B$(\unit{\mm}) & ~~86.9 & ~~86.9 & ~~87.0 & ~~86.9 & ~~87.0 & ~~86.94 \\
        $x_D$(\unit{\mm}) & 107.0 & 106.9 & 107.0 & 107.1 & 107.1 & 107.02 \\
        \bottomrule
    \end{tabularx}
\end{table}
\subsection{不确定度分析}
    \subsubsection{A 类不确定度分量的评估}
    对一个随机变量 $x$ 进行了 $n$ 次重复测量,其算术平均值 $\bar{x}$ 作为对被测量 $x$ 的最佳统计评估值,平均值的标准偏差 $s(\bar{x})$ 作为实验结果的标准不确定度,自由度 $\nu=n-1$。计算公式为\\
    \begin{align}
        \bar{x}&=\frac{1}{n}\sum_{i=1}^{n}x_i \label{exp:amean} \\
        s(\bar{x})&=\sqrt{\frac{1}{n(n-1)}\sum_{i=1}^{n}(x_i-\bar{x})^2} \label{exp:astd}
    \end{align}

    \subsubsection{B 类不确定度分量的评估}
    对一个非重复测量得到的量 x 进行不确定度分析时,评估者需根据已获得的信息人为假定其服从某种概率分布来评估其方差 $u^2(x)$ 或标准不确定度 $u(x)$。具体步骤是,首先根据量 $x$ 的变化范围确定其半宽度 $a$,然后根据假定的概率分布计算标准不确定度。\\
    \begin{equation}
        u(x) = \frac a k \label{exp:uncertaintyB}
    \end{equation}
    在 100\% 和 95\% 置信概率下,矩形分布时的包含因子 $k$ 值分别为 $\sqrt{3}$ 和 $1.65$,三角形分布时的包含因子 $k$ 值分别为 $\sqrt{6}$ 和 $1.90$,正态分布时的包含因子 $k$ 值分别为 $3$ 和 $1.96$。

    \subsubsection{直接测量量的合成标准不确定度}
    直接测量量的合成标准不确定度 $u_c(x)$ 由其 A 类标准不确定度分量 $s(\bar{x})$ 与其 B 类标准不确定度分量 $u(x)$ 采用方和根法合成,即\\
    \begin{equation}
        u_c(x) = \sqrt{s^2(\bar{x})+u^2(x)} \label{exp:uncertaintyCombine}
    \end{equation}
    合成标准不确定度 $u_c$ 的自由度称为有效自由度,用 $\nu_\text{eff}$ 表示。当各分量间相互独立且合成量接近正态分布或 t 分布时,有效自由度 $\nu_\text{eff}$ 可以由下面的韦尔奇-萨特韦特(Welch-Satterthwaite)公式计算\\
    \begin{equation}
        \nu_\text{eff}=\frac{u_c^4(x)}{\displaystyle\frac{s^4(x)}{\nu_\text{A}}+\frac{u^4(x)}{\nu_\text{B}}} \label{exp:validFreedom}
    \end{equation}
    式中 $\nu_\text{A}$ 和 $\nu_\text{B}$ 分别是 A 类和 B 类标准不确定度分量对应的自由度数。如果 $\nu_\text{eff}$ 不是整数,则去掉小数部分取整,即将 $\nu_\text{eff}$ 取为一个不大于 $\nu_\text{eff}$ 本身的整数。

    \subsubsection{标准不确定度的传播规律}
    标准不确定度传播规律的数学基础是全微分。假设间接测量量 $y$ 与直接测量量 $x_1$, $x_2$, $x_3$, $\cdots$, $x_n$ 满足函数关系 $y = f(x_1,x_2,x_3,\cdots,x_n)$,且各个直接测量量 $x_1$, $x_2$, $x_3$, $\cdots$, $x_n$ 是相互独立的。对于以标准偏差表示的标准不确定度,需以方和根的形式求和,因此,当各个直接测量量依次有 $u(x_1)$, $u(x_2)$, $u(x_3)$, $\cdots$, $x_n$ 的标准不确定度时,间接测量量 $y$ 的标准不确定度 $u_c(y)$ 可以表示为\\
    \begin{equation}
        u_c^2(y)=\sum_{i=1}^{n}\left[\frac{\partial f}{\partial x_i}\right]^{2}u^2\left(x_{i}\right)=\sum_{i=1}^{n}c_i^2 u^2\left(x_{i}\right) \label{exp:uncertaintyPropagation}
    \end{equation}
    间接测量量 $y$ 的有效自由度由下式计算\\
    \begin{equation}
        \nu_\text{eff}=\frac{u_c^4(y)}{\displaystyle\sum_{i=1}^{n}\frac{c_i u^4(x_i)}{\nu_i}} \label{exp:validFreedom2}
    \end{equation}

    当需要评定扩展不确定度 $U$ 时,可根据合成标准不确定度的有效自由度 $\nu_\text{eff}$ 和给定的置信概率(譬如 95\%),通过查 t 分布表得出包含因子 $k$,进而给出扩展不确定度 $U = k u_c$。

\subsection{间接测量量计算公式\label{sss:formula}}
\begin{align*}
    \text{超声波频率}\; f&=\frac 4 t \\
    \text{直探头延迟}\; t_s&=2t_1-t_2 \\
    \text{纵波声速}\; c_L&=\frac{2H}{t_2-t_1} \\
    \text{纵波波长}\; \lambda_L&=\frac{c_L}{f} \\
    \text{斜探头延迟}\; t_i&=2t_3-t_4 \\
    \text{横波声速}\; c_S&=\frac{2(R_2-R_1)}{t_4-t_3} \\
    \text{横波波长}\; \lambda_S&=\frac{c_S}{f} \\
    \text{斜探头前沿距离}\; L_0&=R_2-L \\
    \text{斜探头折射角}\; \beta&=\tan^{-1}\left(\frac{L_{B1}-L_{A1}-L_{AB}}{H_{AB}}\right) \\
    \text{直探头声束扩散角}\; \theta_s&=2\tan^{-1}\frac{\left|x_2-x_1\right|}{2H_B} \\
    \text{斜探头声束扩散角}\; \theta_i&=2\tan^{-1}\left(\frac{\left|x_4-x_3\right|}{2H_B}\cos^2 \beta \right) \\ 
    \text{缺陷深度}\; H_C&=c_L\frac{t_c-t_s}{2} \\
    \text{试块内超声声速}\; c&=\frac{2H_{AB}}{(t_B-t_A)\cos \beta} \\
    \text{探头延迟}\; t_0&=t_B-\frac{2H_{B}}{c\cos \beta} \\
    \text{前沿距离}\; L_0^{'}&=H_B\tan\beta-(x_B-L_B) \\
    \text{缺陷 D 的垂直深度}\; H_D&=\frac{c(t_D-t_0)\cos\beta}{2} \\
    \text{缺陷 D 的水平距离}\; L_D&=x_D+L_0^{'}-H_D\tan\beta
\end{align*}
\subsection{不确定度的计算}
利用 \ref{sss:data} 的数据以及 \ref{sss:uncertaintyFormula}、\ref{sss:formula} 中的公式可求得各个测量量的不确定度和自由度,如表 \ref{tab:uncertaintyResult} 所示,部分表头省略了不确定度四字。
\begin{longtable}{@{}>{\hfil}p{0.18\textwidth-1em}<{\hfil}@{\hspace*{1mm}}%
    S[table-format=3.6]S[table-format=2.7]S[table-format=2.6]%
    S[table-format=2.6]S S[table-format=2.6]}
    \caption{各测量量的不确定度\label{tab:uncertaintyResult}} \\ \toprule
    测量量 & {平均值} & {A 类} & {B 类} & {合成} & {有效自由度} & {扩展} \\ \midrule
    \endfirsthead

    \multicolumn{7}{r}{\small 表 \ref{tab:uncertaintyResult} (续)} \\ \toprule
    测量量 & {平均值} & {A 类} & {B 类} & {合成} & {有效自由度} & {扩展} \\ \midrule
    \endhead

    \bottomrule
    \endfoot

    \bottomrule
    \endlastfoot

    $L_A$(\unit{\mm}) & 19.84 & 0.02 & 0.0577 & 0.0611 & 435.6 & 0.12 \\ 
    $L_B$(\unit{\mm}) & 50.02 & 0.0163 & 0.0577 & 0.06 & 911.2 & 0.118 \\ 
    $L_{AB}$(\unit{\mm}) & 30.18 & 0.0306 & 0.0577 & 0.0653 & 104.5 & 0.13 \\ 
    $H_A$(\unit{\mm}) & 20.3 & 0.0258 & 0.0577 & 0.0632 & 180 & 0.125 \\ 
    $H_B$(\unit{\mm}) & 50.36 & 0.02 & 0.0577 & 0.0611 & 435.6 & 0.12 \\ 
    $H_{AB}$(\unit{\mm}) & 30.06 & 0.0416 & 0.0577 & 0.0712 & 42.7 & 0.144 \\[1em]

    $t$(\unit{\us}) & 1.49 & 0.00447 & 0.00577 & 0.0073 & 35.6 & 0.0148 \\ 
    $f$(\unit{\MHz}) & 2.68456 & 0.00806 & 0.0104 & 0.0132 & 35.6 & 0.0267 \\[1em]

    $t_1$(\unit{\us}) & 15.238 & 0.00542 & 0.00577 & 0.00792 & 22.8 & 0.0164 \\ 
    $t_2$(\unit{\us}) & 29.678 & 0.00702 & 0.00577 & 0.00909 & 14 & 0.0195 \\ 
    $t_s$(\unit{\us}) & 0.798 & 0.0129 & 0.0129 & 0.0183 & 34.3 & 0.0371 \\ 
    $c_L$(\unit{\mm\per\us}) & 6.23269 & 0.00383 & 0.00352 & 0.0052 & 32.1 & 0.0106 \\ 
    $\lambda_L$(\unit{\m}) & 2.32168 & 0.00711 & 0.00909 & 0.0115 & 37.6 & 0.0234 \\[1em]

    $t_3$(\unit{\us}) & 25.262 & 0.00163 & 0.00577 & 0.006 & 911.3 & 0.0118 \\ 
    $t_4$(\unit{\us}) & 44.278 & 0.00163 & 0.00577 & 0.006 & 911.3 & 0.0118 \\ 
    $t_i$(\unit{\us}) & 6.246 & 0.00365 & 0.0129 & 0.0134 & 1340.1 & 0.0263 \\ 
    $c_S$(\unit{\mm\per\us}) & 3.15524 & 0.000383 & 0.00135 & 0.00141 & 1822.5 & 0.00276 \\ 
    $\lambda_S$(\unit{\m}) & 1.17533 & 0.00353 & 0.00458 & 0.00578 & 36.1 & 0.0117 \\[1em]

    $L$(\unit{\mm}) & 42.3 & 0.0447 & 0.0577 & 0.073 & 35.6 & 0.148 \\ 
    $L_0$(\unit{\mm}) & 17.7 & 0.0447 & 0.0577 & 0.073 & 35.6 & 0.148 \\[1em]

    $L_{A1}$(\unit{\mm}) & 27.48 & 0.0163 & 0.0577 & 0.06 & 911.2 & 0.118 \\ 
    $L_{B1}$(\unit{\mm}) & 87.12 & 0.0306 & 0.0577 & 0.0653 & 104.5 & 0.13 \\ 
    $\beta$(${}^\circ$) & 44.4224 & 0.0599 & 0.112 & 0.127 & 315.1 & 0.249 \\[1em]

    $x_1$(\unit{\mm}) & 44.24 & 0.0416 & 0.0577 & 0.0712 & 42.7 & 0.144 \\ 
    $x_2$(\unit{\mm}) & 57.02 & 0.0306 & 0.0577 & 0.0653 & 104.5 & 0.13 \\ 
    $x_3$(\unit{\mm}) & 82.96 & 0.02 & 0.0577 & 0.0611 & 435.6 & 0.12 \\ 
    $x_4$(\unit{\mm}) & 94.1 & 0.0258 & 0.0577 & 0.0632 & 180 & 0.125 \\ 
    $\theta_s$(${}^\circ$) & 14.4628 & 0.0581 & 0.0929 & 0.11 & 118.2 & 0.217 \\ 
    $\theta_i$(${}^\circ$) & 6.45803 & 0.0232 & 0.0538 & 0.0586 & 759.9 & 0.115 \\[1em]

    $t_C$(\unit{\us}) & 15.376 & 0.002 & 0.00577 & 0.00611 & 435.6 & 0.012 \\ 
    $H_C$(\unit{\mm}) & 45.4301 & 0.0178 & 0.0253 & 0.031 & 57.8 & 0.062 \\[1em]

    $t_A$(\unit{\us}) & 25.404 & 0.00103 & 0.00577 & 0.00587 & 5200.3 & 0.0115 \\ 
    $t_B$(\unit{\us}) & 51.5512 & 0.00122 & 0.00577 & 0.0059 & 2719.4 & 0.0116 \\ 
    $t_D$(\unit{\us}) & 31.0688 & 1.8 & 0.00577 & 1.8 & 5 & 4.62 \\[2mm]

    $x_A$(\unit{\mm}) & 28.9 & 0.0632 & 0.0577 & 0.0856 & 16.8 & 0.181 \\ 
    $x_B$(\unit{\mm}) & 86.94 & 0.02 & 0.0577 & 0.0611 & 435.6 & 0.12 \\ 
    $x_D$(\unit{\mm}) & 107.02 & 0.0306 & 0.0577 & 0.0653 & 104.5 & 0.13 \\[2mm]

    $c$(\unit{\mm\per\us}) & 3.2194 & 0.00337 & 0.00629 & 0.00714 & 311.6 & 0.014 \\ 
    $t_0$(\unit{\us}) & 7.74638 & 0.0631 & 0.0985 & 0.117 & 68.8 & 0.233 \\ 
    $L_0^{'}$(\unit{\mm}) & 12.4348 & 0.108 & 0.217 & 0.242 & 471 & 0.476 \\ 
    $H_D$(\unit{\mm}) & 26.8125 & 2.06 & 0.0738 & 2.07 & 5 & 5.31 \\ 
    $L_D$(\unit{\mm}) & 93.1775 & 2.02 & 0.127 & 2.03 & 5 & 5.2 \\
\end{longtable}
\section{结果分析}
\end{document}