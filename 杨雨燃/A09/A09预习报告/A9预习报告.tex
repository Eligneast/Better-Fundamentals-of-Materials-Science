\documentclass[a4paper,utf8]{article}
\usepackage[heading,fancyhdr]{ctex}
\usepackage{amsmath,amssymb,geometry,lastpage,ulem}
\usepackage{array,tabularx}
\usepackage{siunitx}
\usepackage{graphicx}
\lineskiplimit=1pt
\lineskip=3pt
\geometry{
    top=25.4mm, 
    left=25mm, 
    right=25mm, 
    bottom=25mm,
    headsep=5.9mm,
}
\ctexset{
    section = {format+=\raggedright}
}
\newcommand{\fgref}[1]{图~\ref{#1} }
\newcommand{\seqref}[1]{式~(\ref{#1})}
\pagestyle{fancy}
\fancyhf{} \fancyhead[C]{材料科学基础实验} \fancyfoot[C]{\thepage\;/\;\pageref{LastPage}}
\begin{document}
\begin{center}
    {\mbox{}\\[7em]\zihao{2}\bfseries\songti
    材料科学基础实验预习报告}\\[34mm]
    {\zihao{-3}\bfseries\songti
    实验名称:\uline{\hfill\mbox{Sn-Bi 合金相图的测定阻}\hfill} \\[2.9mm]
    学\quad 号:\uline{\makebox[25mm]{22301070}}\hfill
    姓\quad 名:\uline{\makebox[25mm]{杨雨燃}}\hfill
    班\quad 级:\uline{\makebox[25mm]{22材物}} \\[2.9mm]
    合作者:\uline{\makebox[25mm]{}}\enspace~
    桌\quad 号:\uline{\makebox[25mm]{}}\hfill\mbox{}\\[2.9mm]
    指导教师:\uline{\makebox[30mm]{艾斌}}\hfill\mbox{} \\[2.9mm]
    实验日期:\uline{\makebox[30mm]{}}\hfill\mbox{} \\[58.7mm]
    }
\end{center}

\newpage

\section*{【实验目的】}
    \begin{enumerate}
        \item 学会用热分析法测绘 Sn-Bi 合金相图;
        \item 了解纯金属和二元合金步冷曲线形状的差异;
        \item 学会从步冷曲线上确定相变点温度的方法;
        \item 学会根据实测的步冷曲线绘制相图
    \end{enumerate}
\section*{【实验原理】}%简单描述,含必要的公式和附图;
相图简介: 相图描述了系统中相与温度、成分、压强之间的关系,在材料热力学平衡条件下反映了各种相的平衡关系。对于凝聚态系统,相图通常以温度和合金成分为坐标轴。

相图建立的关键: 建立相图的关键在于准确测定各成分合金的相变临界点,这些临界点反映了物质结构状态发生本质变化的相变点。测定方法包括热分析法、热膨胀法、电阻测量法、金相法、X射线衍射分析法等,需要多种方法配合使用。

热分析法测绘相图的一般方法: 以热分析法为例,首先配制一系列不同组分比例的混合物,然后加热熔融成均匀液相,缓慢冷却并记录温度与时间的关系曲线,称为冷却曲线或步冷曲线。根据曲线上的转折点确定相变临界点,绘制二元相图。

相图的解读: 对于纯金属,冷却曲线上会出现一个平台期,表示材料在恒温下凝固;而对于合金,曲线上没有平台,而是出现两次转折点,分别表示凝固的开始温度和凝固的终结温度。根据这些临界点在相图上绘制出各相的平衡相区域。    

用热分析法测量Cu-Ni合金并测绘二元合金相图的一般方法为:

\begin{enumerate}
    \item 按质量分数配制一系列不同组分比例的有代表性的Cu-Ni混合物并分别将所配制的Cu—Ni混合物、纯Cu和纯Ni样品加热熔化成单一、均匀的液相,然后让各样品缓慢冷却,并每隔一定时间读取一次各样品的温度,由此可得到各样品的冷却曲线或步冷曲线。
    \item 对于纯Cu或纯Ni,由于不涉及相变,液相的温度以均匀的速率下降。当继续冷却,纯金属的冷却曲线将出现一个温度保持不变的平台期,平台对应的温度为材料的凝固温度。平台期的起点表示液相中开始有固相析出,终点表示液相刚好全部凝固为单一固相。
    \item 对于一定组成的Cu-Ni混合物,随着温度的降低,液相的温度不断降低。当温度达到相变温度时,固相(Cu—Ni固溶体)开始从液相中析出,凝固释放相变潜热,使体系降温的速率变慢,步冷曲线的斜率发生变化而出现第一个拐点,对应于凝固开始的温度。随着继续冷却,由相律公式$f=C-P+1$可知,二元合金凝固(液、固两相共存)时体系的自由度为1,这意味着随着冷却的进行,温度会继续下降。当液相全部凝固为固相时,由于没有新的相变潜热释放,步冷曲线的斜率将再次改变而出现第二个拐点,即凝固结束的温度。随着继续冷却,体系的温度也将以另一速率下降。不同于纯金属的步冷曲线,合金的步冷曲线上没有平台,而是存在两次转折。
    \item 由步冷曲线测定的每个相变临界点在以合金成分为$x$轴、温度为$y$轴的二元系相图中都分别对应一个点,将所有意义相同的临界点(凝固的起始点或终止点)连接起来就得到了Cu—Ni合金相图。
\end{enumerate}

\section*{【实验仪器】}%规格及参数
    JX-3D8 金属相图测量装置,计算机。

\section*{【实验过程】} %简述主要过程和实验内容

本实验利用计算机和该软件可以自动采集、实时显示和保存 8 个样品管中试样的步冷曲线。步冷曲线测试结束后,实验人员需要从步冷曲线或步冷曲线数据上寻找和确定相变点(或拐点)温度,后在操作软件的“查看”下拉菜单中选择“相图”出现绘制相图界面。在操作界面左侧相应的数字框中填入拐点温度、成分、平台温度、共晶成分和共晶温度,由软件绘制出相图。

1. 样品准备:

将1-8号样品管依次放好,分别含有纯Sn、10 wt\% Bi、20 wt\% Bi、40 wt\% Bi、50 wt\% Bi、60 wt\% Bi、80 wt\% Bi和纯Bi等纯金属和Sn-Bi合金。样品管是密封的,试样可重复使用,一般情况下不存在样品氧化和向外挥发的问题。需要说明的是,纯Sn的熔点是232℃,而纯Bi的熔点是271.4℃。

2. Sn-Bi合金步冷曲线的测量:

    设置JX-3D8金属相图测量装置的电源开关,并将加热目标温度设为300℃。为确保样品全部熔化,通常目标温度比体系最高熔化温度再升高30℃左右。
    
    控制样品降温速度足够慢,使体系尽可能处于或接近相平衡的状态,以保证步冷曲线上的拐点和水平段的准确测量。控制降温速度通常在5-8℃/min之间,可通过JX-3D8金属相图测量装置的“保温”和“散热”功能进行调节。
    
    使用JX-3D8金属相图测量装置配备的计算机,通过“金属相图(8通道)实验软件”控制计算机和相图测量装置完成步冷曲线数据的自动采集和实时显示。

3. 相变点温度的确定和Sn-Bi合金相图的绘制:
    
    根据步冷曲线确定8种样品的相变点(或拐点)温度,并填写表1。
    
    根据表1的数据绘制Sn-Bi合金相图。

\section*{【实验数据】}

1.气温:\uline{\makebox[25mm]{}} ℃,湿度:\uline{\makebox[25mm]{}} \%


2.8种样品步冷曲线上的拐点温度:

\begin{table}[ht!]
    \centering
    \caption{8种样品步冷曲线上的拐点温度}
    \begin{tabular}{|c|c|c|c|c|c|c|c|c|}
    \hline
        温度 & 纯Sn & 10\% Bi & 20\% Bi & 40\% Bi & 50\% Bi & 60\% Bi & 80\% Bi & 纯Bi \\ \hline
        第一拐点 &  &  &  &  &  &  &  &  \\ \hline
        第二拐点 &  &  &  &  &  &  &  &  \\ \hline
        \end{tabular}
    \end{table}
    
\end{document}