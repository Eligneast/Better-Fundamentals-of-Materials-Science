\documentclass[a4paper,utf8]{article}
\usepackage[heading,fancyhdr]{ctex}
\usepackage{amsmath,amssymb,geometry,lastpage,ulem}
\usepackage{array,tabularx,tabulary,mhchem,xspace}
\usepackage{floatrow,subfig,multirow,bigstrut}
\usepackage{siunitx,graphicx}
\lineskiplimit=1pt
\lineskip=3pt
\geometry{
    top=25.4mm, 
    left=25mm, 
    right=25mm, 
    bottom=25mm,
    headsep=5.9mm,
}
\newcommand{\fgref}[1]{图~\ref{#1}\xspace}
\newcommand{\seqref}[1]{式~(\ref{#1})}
\newcommand{\expinfo}[5]{
    {\zihao{-3}\bfseries\songti
    实验名称:\uline{\hfill\mbox{#1}\hfill} \\[2.9mm]
    学\quad 号:\uline{\makebox[25mm]{#2}}\hfill
    姓\quad 名:\uline{\makebox[25mm]{#3}}\hfill
    班\quad 级:\uline{\makebox[25mm]{#4}} \\[2.9mm]
    合作者:\uline{\makebox[25mm]{无}} \hfill
    桌\quad号:\uline{\makebox[25mm]{}}\hfill\makebox[25mm+4em]{}\\[2.9mm]
    指导教师:\uline{\makebox[30mm]{#5}}\hfill\mbox{} \\[2.9mm]
    实验日期:\uline{\makebox[30mm]{}}\hfill\mbox{} \\[58.7mm]
    }
}%\expinfo{实验名称}{学号}{姓名}{班级}{指导教师}
\pagestyle{fancy}
\fancyhf{} \fancyhead[C]{材料科学基础实验} \fancyfoot[C]{\thepage~/~\pageref{LastPage}}
\begin{document}
\begin{center}
    {\mbox{}\\[7em]\zihao{2}\bfseries\songti%
    材料科学基础实验预习报告}\\[34mm]
    \expinfo{碳钢退火、正火后的组织观察与硬度分析}{22301077}{张蕴东}{22高分子}{杨玉华}
    \pointingbox
\end{center}
\newpage
\section{实验目的}
    \begin{enumerate}
        \item 了解碳钢的退火、正火过程。
        \item 观察和研究碳钢经不同退火处理、正火处理后显微组织的特点,分析热处理工艺对其组织与硬度的影响,并了解退火、正火的应用领域。
    \end{enumerate}
\section{实验原理}%简单描述,含必要的公式和附图;
    热处理是一种很重要的热加工工艺方法。热处理的主要目的是改变钢的性能,其中包括使用性能及工艺性能。绝大多数重要的机械零件在制备过程中都要经过热处理,其目的就是把工件加热到一定温度,然后根据不同的要求采取不同的保温时间、冷却速度,使金属零件中的结构产生预期的变化,从而使零件具有不同的机械性能。\par
    热处理包括四个因素:加热速度、最高温度、加热时间(通常将工件升温和保温所需时间算在一起,统称为加热时间)和冷却速度。根据热处理工艺的使用和进行方式不同,热处理一般可分为下列四种类型:退火、正火、淬火、回火。\par
    \subsection{退火}
        退火是将工件加热到适当温度,根据材料和工件尺寸采用不同的保温时间,然后进行缓慢冷却(冷却速度最慢),目的是使工件内部组织达到或接近平衡状态,获得良好的工艺性能和使用性能,或者为进一步淬火过程作组织准备。退火的工艺方法有很多种,其中包括完全退火、不完全退火、球化退火和去应力退火等。在实际生产中,经常采用退火作为预备热处理工序,安排在锻造、铸造等热加工之后,切削加工之前,为下一道工序作组织和性能上的准备。
        \subsubsection{完全退火}
            将亚共析钢件加热到 $A_3+(30\sim 50)~\unit{\degreeCelsius}$($A_3$ 线是钢 $\alpha$ 固溶体转变为 $\gamma$ 固溶体之临界温度),保温一段时间,然后缓慢地随炉冷却的工艺方法。
        \subsubsection{不完全退火}
            将亚共析钢加热到 $A_1 \sim A_3~\unit{\degreeCelsius}$($A_1$ 为共析转变温度),过共析钢加热到 $A_1 \sim A_{\textrm{cm}}~\unit{\degreeCelsius}$(碳在奥氏体中的溶解度曲线所对应的温度为 $A_{\textrm{cm}}$ 温度),保温后缓慢冷却的方法。
        \subsubsection{球化退火}
            将过共析钢件加热到 $A_1+20~\unit{\degreeCelsius}$ 左右,保温一定时间后以适当的方式冷却使钢中的碳化物球状化的工艺方法。
        \subsubsection{去应力退火}
            将钢件加热到相变点 $A_1$ 以下的某一温度,保温一定时间后缓慢冷却的工艺方法。其目的是为了消除由于冷热加工所产生的残余应力。一般碳钢和低合金钢加热温度为 $550 \sim 600~\unit{\degreeCelsius}$,而高合金钢一般为 $600 \sim 700~\unit{\degreeCelsius}$,保温一定时间(一般近 \SI{3}{\minute/\milli\metre}),然后随炉缓慢冷却($\le \SI{100}{\degreeCelsius/\hour}$)到\SI{200}{\degreeCelsius}出炉。对于铸铁件一般加热到 $500 \sim 550~\unit{\degreeCelsius}$,不能超过 \SI{550}{\degreeCelsius},因为超过 \SI{550}{\degreeCelsius} 以后铸铁中的珠光体要发生石墨化。
        \subsubsection{扩散退火}
            又称均匀化退火,用于合金钢锭和铸件,以消除枝晶偏析,使成分均匀化。扩散退火是把铸锭或铸件加热到略低于固相线以下某一温度,通常为 $A_3$ 或 $A_{\textrm{cm}} + (150 \sim 300)~\unit{\degreeCelsius}$,长时间保温后随炉缓慢冷却的一种热处理工艺方法。一般碳钢采用 $1100 \sim 1200~\unit{\degreeCelsius}$,合金钢采用 $1200 \sim 1300~\unit{\degreeCelsius}$,保温时间为 $10 \sim 15~\unit{\hour}$。
    \subsection{正火}
        正火是退火的特殊形式,其与一般退火所不同之处是试样在具有稍大的冷却速度的空气中进行冷却。\par
        正火加热温度选择:正火则是将钢材加热到 $A_3$ 或 $A_{\textrm{cm}} + (30 \sim 50)~\unit{\degreeCelsius}$,保持一定时间后在空气中进行冷却。一般亚共析钢加热至$A_3 + (30 \sim 50)~\unit{\degreeCelsius}$;过共析钢加热至 $A_{\textrm{cm}} + (30 \sim 50)~\unit{\degreeCelsius}$,即加热到奥氏体单相区。退火和正火加热温度范围选择见\fgref{fig:1}。
        \begin{figure}[!ht]
        \begin{floatrow}\centering
            \ffigbox[60mm]{\caption{退火和正火的加热温度范围\label{fig:1}}}{\includegraphics[height=40mm]{fig1.jpg}}
            \ffigbox[\Xhsize]{\caption{碳钢热处理后的金相组织($450\times$)。(a) T12 钢球化退火,(b) 45 钢正火\label{fig:2}}}{\includegraphics[height=40mm]{fig2.jpg}}
        \end{floatrow}
        \end{figure}
    \subsection{退火与正火保温时间的确定}
        在装炉量不太大时,可用下式计算保温时间
        \begin{equation}
            \tau = K D \label{equ:1}
        \end{equation}
        式中,$K$ 为加热系数,一般 $K=1.5 \sim \SI{2.0}{\minute/\milli\metre}$;$D$ 为工件有效尺寸。合金钢的保温时间比碳钢长一些,工件越大,装炉量越多,保温时间也越长。
    \subsection{碳钢退火、正火后的显微组织}
        亚共析碳钢一般采用完全退火,经退火后可得接近于平衡状态的组织,即铁素体加珠光体。过共析碳素工具钢则多采用球化退火,获得在铁素体基体上均匀分布着的粒状渗碳体的组织,称为球状珠光体或球化体。如 T12 钢经球化退火后组织为球状珠光体。二次渗碳体和珠光体中的渗碳体都呈球状(或粒状),如\fgref{fig:2}(a)所示,在铁素体基体上分布的颗粒状 \ce{Fe3C},铁素体基体上白色小颗粒为 \ce{Fe3C}。\par
        碳钢正火后的组织比退火的细,并且亚共析钢的组织中细珠光体(索氏体)的质量分数比退火组织中的多,并随着碳质量分数的增加而增加。45 钢正火组织为铁素体 $+$ 索氏体,如\fgref{fig:2}(b)。其中白色条状为铁素体,沿晶界析出;黑色块状为索氏体。正火冷速快,铁素体得不到充分析出,质量分数少,进行共析反应的奥氏体增多,析出的珠光体多而细。
    \subsection{注意事项}
    \begin{enumerate}
        \item 本实验加热为高温马弗炉,在放、取试样时一定要注意安全。
        \item 往炉中放、取试样必须使用夹钳,夹钳必须擦干,不得沾有油和水。
        \item 热处理后的试样均要用砂纸打磨掉表面黑色氧化皮后再测定硬度值
    \end{enumerate}
\section{实验仪器}%规格及参数
    箱式电阻加热炉,洛氏硬度计,砂纸,抛光机,金相显微镜。热处理试样:45 钢及 T12 钢。
\section{实验过程}%简述主要过程和实验内容
    \begin{enumerate}
        \item 每4人一组,领取45钢(2个)、T8(1个)及T12钢(1个)试样一套,每组共同完成一套实验(对应下表中相应的热处理工艺方法)。
        \begin{table}[!ht]\centering
            \caption{试样的热处理工艺}
            \newcounter{sample} \newcommand{\Sam}{\stepcounter{sample}\thesample}
            \begin{tabular}{|c|c|c|c|c|}\hline
                试样号码 & 钢号 & 热处理工艺 & 浸蚀剂 & 建议放大倍数 \\ \hline
                \Sam & 45 & 完全退火 & 4\% 硝酸酒精 & 200 $\sim$ 500 \\ \hline
                \Sam & 45 & 正火 & 4\% 硝酸酒精 & 200 $\sim$ 500 \\ \hline
                \Sam & T8 & 正火 & 4\% 硝酸酒精 & 200 $\sim$ 500 \\ \hline
                \Sam & T12 & 球化退火 & 4\% 硝酸酒精 & 200 $\sim$ 500 \\ \hline
            \end{tabular}
        \end{table}
        \item 制定热处理工艺参数,可参考以下工艺参数。
        \begin{enumerate}
            \item 45 钢完全退火工艺:加热温度为 $860 \pm \SI{10}{\degreeCelsius}$,根据试样有效尺寸计算保温时间,保温后炉冷到\SI{500}{\degreeCelsius}左右出炉空冷。
            \item 45 钢正火工艺:加热温度为 $860 \pm \SI{10}{\degreeCelsius}$,根据试样有效尺寸计算保温时间,保温后出炉空冷。
            \item T8 钢正火工艺:加热温度为 $820 \pm \SI{10}{\degreeCelsius}$,根据试样有效尺寸计算保温时间,保温后出炉空冷。
            \item T12 钢球化退火工艺:加热温度为 $760 \pm \SI{10}{\degreeCelsius}$,根据试样有效尺寸计算保温时间(约 40 分钟),保温后随炉冷却到 \SI{680}{\degreeCelsius} 保温 40 分钟,随后炉冷到 \SI{500}{\degreeCelsius} 出炉空冷。
        \end{enumerate}
        \item 利用硬度计对所有热处理后的试样进行硬度测试,每个试样至少三个试验点,再取一个平均值,分析热处理工艺对其硬度的影响。(硬度测试须在金相磨制观察前完成)
        \item 根据拟定的热处理工艺对试样进行相应的热处理工艺处理,然后利用金相砂纸对热处理后的\textbf{试样进行磨制、抛光},并用 4\% 的硝酸酒精进行腐蚀制得金相试样。利用金相显微镜对其进行\textbf{显微组织观察},分析热处理工艺对其组织的影响。
        \item 实验结束后,汇总各小组实验数据,根据实验数据分析冷却方法对碳钢性能(硬度)的影响,并阐明硬度变化的原因。
    \end{enumerate}

\section{实验结果}
    本人处理的是 45 钢退火工艺。\par
    \subsection{硬度数据}
        \begin{table}[!ht]\centering
            \caption{不同热处理试样的硬度值}
            \begin{tabularx}{0.8\textwidth}{c*{4}{C}}\toprule
                材料及热处理状态 & \multicolumn{3}{c}{测得硬度数据} & 平均值 \bigstrut \\ \midrule
                20 钢正火(HRB) & 78.2 & 79.0 & 77.8 & 78.3 \bigstrut \\
                20 钢退火(HRB) & 46.1 & 43.6 & 47.0 & 45.6 \bigstrut \\ 
                45 钢正火(HRB) & 94.4 & 92.0 & 94.7 & 93.7 \bigstrut \\
                45 钢退火(HRB) & 88.5 & 86.0 & 86.5 & 87.0 \bigstrut \\
                T8 钢正火(HRC) & 32.4 & 34.1 & 32.6 & 33.0 \bigstrut \\
                T8 钢退火(HRB) & 83.2 & 82.0 & 83.8 & 83.0 \bigstrut \\
                T12 钢球化退火(HRC) & \multicolumn{4}{c}{本组无人做该工艺} \bigstrut \\
                \bottomrule
            \end{tabularx}
        \end{table}
    \subsection{试样金相图}
        见下一页
        \begin{figure}[!ht]
            \subfloat[20 钢正火处理 500x]{\includegraphics[width=0.40\textwidth]{exp_img/fig20a.jpg}\label{expfig:fig20a}} \hspace{30pt}
            \subfloat[20 钢退火处理 500x]{\includegraphics[width=0.40\textwidth]{exp_img/fig20b.jpg}\label{expfig:fig20b}} \\
            \subfloat[45 钢正火处理 500x]{\includegraphics[width=0.40\textwidth]{exp_img/fig45a.jpg}\label{expfig:fig45a}} \hspace{30pt}
            \subfloat[45 钢退火处理 500x]{\includegraphics[width=0.40\textwidth]{exp_img/fig45b.jpg}\label{expfig:fig45b}} \\
            \subfloat[T8 钢正火处理 500x]{\includegraphics[width=0.40\textwidth]{exp_img/figT8a.jpg}\label{expfig:figT8a}} \hspace{30pt}
            \subfloat[T8 钢退火处理 500x]{\includegraphics[width=0.40\textwidth]{exp_img/figT8b.jpg}\label{expfig:figT8b}} 
        \end{figure}
    \subsection{试样硬度比较}
        \subsubsection{不同热处理工艺}
            对于不同热处理工艺而言,我们可以发现,正火后硬度都要大于退火后硬度,这一点在20钢、45钢、T8钢上都适用。这固然是正火工艺要求在空气中冷却,与退火工艺在炉内降温相比冷却速率更快而导致的。一方面,降温曲线斜率的不同将会得到不同的组织,通常正火后的钢材中残余奥氏体和碳化物可以充分转变为珠光体或者贝氏体组织等硬度较高的组织;另一方面,降温速率大意味着动态过冷度大,促进形核,将会得到更加细的晶粒(45钢的金相图可以证明这一点),根据细晶强化的原理,此时钢材的硬度也将得到提升。
        \subsubsection{不同的钢材}
            在铁碳相图上,20 钢通常为低碳钢,完全退火过程中,在 860℃ 左右进行保温后,钢材中的残余奥氏体将会转变为渗碳体和珠光体组织,珠光体的形成速度较慢。因此 20 钢的冷却曲线可能比较平缓,导致组织结构中存在较多的软相,使 20 钢的硬度可能相对较低。\par
            45 钢通常含有较高的碳含量,因此其在完全退火过程中,相较于 20 钢,珠光体的形成速度可能会更快,使 45 钢的硬度相对更高。\par
            T8 钢的含碳量最高,珠光体的形成速度更快,自然 T8 钢应当是这之中最硬的。
    \subsection{金相图比较}
        \subsubsection{20 钢}
            对于正火处理,推测黑色部分应为珠光体,高光部分应为铁素体,其珠光体尺寸大,分布均匀。但退火处理的图由于该实验者的处理不够完美,只能粗略看出其中黑色部分较多,可能是珠光体、残余奥氏体和碳化物较多。
        \subsubsection{45 钢}
            与 20 钢相似,但 45 钢正火的组织更细,应该是含碳量更高导致的; 45 钢退火的黑色部分相对更多,应该是含碳量更高、珠光体占主导,并且降温速度慢,形成了更粗大的组织。
        \subsubsection{T8 钢}
            与 20 钢/45 钢组织相比,T8 钢出现了明显不同的细条状组织,即索氏体(细珠光体),整体仍然是由于含碳量高,珠光体占主导。
\section{思考题}
    \subsection{45 钢常用的热处理是什么?它们的组织是什么?有何工程应用?}
        \begin{enumerate}
            \item 淬火:将钢加热至适当温度,然后通过油淬、水淬、盐淬(降温速率依次提高)迅速冷却至室温,以获得马氏体组织。淬火后的 45 钢常用于制作要求高强度和硬度的零件,如刀具、轴承和齿轮等。
            \item 回火:将淬火后的钢加热至较低的温度,本次实验是分别在为 200℃、400℃、600℃ 的低温、中温、高温回火一小时,然后空冷。低温、中温、高温回火后的组织分别是回火马氏体、回火屈氏体、回火索氏体。回火后的 45 钢用于要求具有一定韧性的零件,如机械零件、传动轴等。
            \item 退火:将钢加热至适当温度,然后缓慢冷至室温。退火后的组织以铁素体+索氏体为主,具有较低的硬度和较高的韧性,适用于一些需要韧性较高的零件,例如弹簧、螺栓、轴承等,因为退火后的钢材具有较高的韧性和一定的强度,能够承受较大的冲击负荷。
            \item 正火:将钢加热至适当温度,然后保温一段时间,最后在空气中冷却。正火后的组织以珠光体+铁素体为主,具有适中的硬度和韧性,适用于制造对强度和硬度要求较高的零件,如机械零件、传动轴、齿轮等,因为正火后的钢材具有较高的强度和硬度,能够承受较大的静态负荷和磨损。
        \end{enumerate}
    \subsection{退火状态的 45 钢试样分别加热到 600℃$\sim$900℃之间不同的温度后,在水中冷却,其硬度随加热温度如何变化?为什么?}
        硬度应随加热温度的升高而升高。\par
        原因:45 钢的含碳量约为 0.42\%至 0.50\%,属于亚共析钢。随着温度的升高,试样中的奥氏体比例逐渐增加,而奥氏体在水中冷却后会转变为马氏体,从而导致钢材硬度的增加。由于水淬冷却速率极快,奥氏体中的一部分碳原子会固溶在马氏体中,形成强硬的马氏体组织,进而提高了试样的硬度。因此,随着加热温度的升高,试样中马氏体的比例逐渐增加,导致了试样硬度的增加。
    \subsection{T12 钢经球化退火后得到的组织在本质、形态上有什么特点?}
        经参考其他组得到的T12 钢经球化退火后的金相图和数据、结合所学知识:
        \begin{enumerate}
            \item 本质特点:球化退火是一种热处理工艺,其本质是通过控制温度和时间,使钢材中的碳化物均匀分布并形成球状的结构。这种球化结构有利于提高钢材的塑性和韧性,减少应力集中,提高其耐疲劳性能。
            \item 形态特点:经球化退火后,T12 钢的组织呈现出均匀细小的球状碳化物分布。这种球状碳化物分布可以有效地提高钢材的强度和硬度,同时不会降低其韧性。球化退火后的组织形态有助于提高钢材的加工性能和使用寿命,使其在高温和高应力条件下表现更优异。
        \end{enumerate}
\end{document}