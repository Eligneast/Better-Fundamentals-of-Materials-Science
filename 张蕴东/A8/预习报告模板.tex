\documentclass[a4paper,utf8]{article}
\usepackage{graphicx}
\usepackage[heading,fancyhdr]{ctex}
\usepackage{amsmath,amssymb,geometry,ulem}
\usepackage{array,tabularx,tabulary,mhchem,xspace}
\usepackage{floatrow,subfig,multirow,bigstrut}
\usepackage{siunitx,booktabs,longtable,nameref}
\lineskiplimit=1pt
\lineskip=3pt
\geometry{
    top=25.4mm, 
    left=25mm, 
    right=25mm, 
    bottom=25mm,
    headsep=5.9mm,
}
\ctexset{
    chapter = {
        name = {实验,},
        beforeskip = {-23pt}
    }
}
\newcommand{\fgref}[1]{图~\ref{#1}\xspace}
\newcommand{\seqref}[1]{式~(\ref{#1})}
\newcommand{\expinfo}[6][无]{
    {\zihao{-3}\bfseries\songti
    实验名称:\uline{\hfill\mbox{#2}\hfill} \\[2.9mm]
    学\quad 号:\uline{\makebox[25mm]{#3}}\hfill
    姓\quad 名:\uline{\makebox[25mm]{#4}}\hfill
    班\quad 级:\uline{\makebox[25mm]{#5}} \\[2.9mm]
    合作者:\uline{\makebox[25mm]{#1}} \hfill
    桌\quad 号:\uline{\makebox[25mm]{}}\hfill\makebox[25mm+4em]{}\\[2.9mm]
    指导教师:\uline{\makebox[30mm]{#6}}\hfill\mbox{} \\[2.9mm]
    实验日期:\uline{\makebox[30mm]{}}\hfill\mbox{} \\[58.7mm]
    }
}%\expinfo[合作者]{实验名称}{学号}{姓名}{班级}{指导教师}
\newcommand{\pointingbox}{
    {\zihao{4}\bfseries\songti%
    实验考核\\[3mm]
    \extrarowheight=3mm
    \begin{tabularx}{150mm}{|X|X|X|X|X|}\hline
        \hfil 项目 \hfil  & \hfil 实验预习 \hfil & \hfil 实验过程 \hfil & \hfil 分析与讨论 \hfil & \hfil 总评 \hfil \\[3mm] \hline
        \hfil 评价 \hfil &  &  &  &  \\[3mm] \hline
    \end{tabularx}
    }
}
\newcommand{\derivative}[2]{\frac{\mathrm{d} #1}{\mathrm{d} #2}}
\newcommand{\thinking}[2]{\textbf{#1}\\
答:\begin{minipage}[t]{0.85\textwidth}
    #2
\end{minipage}}

\pagestyle{fancy}
\fancyhf{}
%\fancyhead[C]{材料科学基础实验}
%\fancyfoot[C]{\thepage}
\fancyhead[EC]{\leftmark} \fancyhead[OC]{\rightmark}
\fancyhead[EL,OR]{\thepage}
\fancypagestyle{plain}{\renewcommand{\headrulewidth}{0pt}\fancyhf{}}

\newcounter{Rownumber}
\newcommand*{\Rown}{\stepcounter{Rownumber}\theRownumber}
\newcounter{sample}
\newcommand*{\Sam}{\stepcounter{sample}\thesample}
\newcounter{Fignumber}
\newcommand*{\Fign}{\stepcounter{Fignumber}\theFignumber}

\newcommand*{\resetRown}{\setcounter{Rownumber}{0}}
\newcommand{\qrange}[3]{\qtyrange[range-phrase = \text{$\sim$},range-units =single]{#1}{#2}{#3}}
\floatsetup[table]{capposition=top}
\newcolumntype{C}{>{\hfil}X<{\hfil}}
\renewcommand{\Nameref}[1]{\textbf{\ref{#1}~\nameref{#1}}}
\newcommand{\TTR}[0]{\watt\per\m\per\K} %导入导言
\begin{document}
\begin{center}
    {\mbox{}\\[7em]\zihao{2}\bfseries\songti%
    材料科学基础预习报告}\\[34mm]
    \expinfo{磁性材料基础测量}{22301077}{张蕴东}{22高分子}{李继玲}
\end{center}
\newpage
\section{实验目的}
    \begin{itemize}
        \item 掌握测量铁磁材料磁滞、动态磁滞回线和基本磁化曲线的原理和方法,加深对铁磁材料基本磁性参量的理解 
        \item 学习用电子积分器测量磁感应强度
        \item 学会根据磁性材料的磁性曲线确定其矫顽力、剩余磁感强度、饱和磁感强度,磁滞损耗、磁滞损耗等磁化参数
        \item 学习测量磁性材料磁导率 $\mu$ 的 一种方法,了解磁性材料的主要特性
    \end{itemize}
\section{实验原理}%简单描述,含必要的公式和附图;
    铁磁材料应用广泛,从常用的永久磁铁、变压器铁芯到录音、录像、计算机存储用的磁带、磁盘等都采用各种特性的铁磁材料。铁磁材料多数是铁和其他金属元素或者非金属组成的合金以及某些包含铁的氧化物(铁氧体),他们除了具有高的磁导率外,另一个重要的磁性特点就是磁滞。铁磁材料的磁滞回线和磁化曲线表征了磁性材料的基本磁化规律,反应了磁性材料的基本参数。对铁磁材料的应用和研制有重要意义。由于磁性材料的磁化过程很复杂,影响磁性材料磁化特性的因素有很多,如:掺杂、结构、温度等。在多数场合无法用解析式来定量描述 H—B 之间的关系,只能通过实验测定它。本实验中,将待测的磁性材料做成闭合环状,上面均匀地绕两组线圈。给其中一组线圈通电流 I ,使其产生强度为 H 的磁化场,这组线圈称为初级线圈。当初级线圈中的电流发生变化时,在另外一组线圈即次级线圈中,将产生感应电动势 $\varepsilon$ ,用电子积分器测出 $\varepsilon$ ,经计算可以得到 B ,根据 H—B 的对应关系可以绘出它们的曲线。
    \subsection{初始磁化曲线}
        研究磁性材料的磁化规律,通常是通过测量磁化场的磁感强度H与磁感应强度 B 的关系来进行的。磁化曲线也叫 B-H 曲线,即表示物质中的磁场强度H与所感应的磁感应强度B之间关系的曲线。铁磁材料的磁化过程非常复杂,B 和 H 的关系如图1 所示。将处在未磁化状态的磁性材料(H=0、B=0)加以磁化,当逐渐增加磁场强度 H 时,磁感强度 B 也随之增加而非线性增加,经过一段急剧增加的过程后又缓慢下来,当 H 增大到一定值 $H_s$ 后, B 增加十分缓慢或者基本不再增加,这时磁化就达到了磁饱和。到达磁饱和的 $H_s$ 和 $B_s$ 分别称为饱和磁场强度和饱和磁感应强度(对应图1-1中的a点)。从未磁化到饱和磁化,H 和 B 对应的关系曲线称为初始磁化曲线或者起始磁化曲线。如图1-1 中 Oa所示。
        \begin{figure}
            
        \end{figure}
    \subsection{磁滞回线}
        磁性材料还有一个重要的特性,那就是磁滞。磁滞是指 B 的变化滞后于 H 的变化。如图1-1所示,当磁性材料的磁化达到饱和之后,如果使 H 单调减小,B 不沿原路沿aO返回, 而是沿一条新的路线(沿abc)下降。当 H 减小到 0,B 并不为 0,而是到达 $B_r$ ,说明此时铁磁材料中仍然保留有一定的磁性,这种现象称为磁滞效应, $B_r$ 称为剩余磁感强度,简称剩磁。要消除剩磁使 B=0 就得加一个反向的磁化场,直到反向磁化场到达 $H_c$ , B 才恢复为零。$H_c$ 被称为矫顽力。如果继续增加反向的磁化场使H继续增加,铁磁材料将被反向磁化直至达到反向饱和,此后减小磁化场 H 直至 0,再沿正方向增加 H 至饱和,曲线回到 a 点。由此得到一条 H—B 的闭合曲线,这条关于原点对称的闭合曲线,称为该材料的磁滞回线。如图1-1所示。实验表明,经过反复的磁化后,铁磁材料达到稳定的磁化状态,B-H 的量值关系形成一个稳定的闭合“磁滞回线”,通常以这条曲线来表示铁磁材料的磁化性质。
    \subsection{基本磁化曲线}
        当从初始状态(H=0, B=0)开始周期性地改变磁场强度H的幅值时,在磁场由弱变强单调增加过程中,可以得到一系列面积由小到大的稳定的磁滞回线,如图1-2所示,其中最大面积的磁滞回线称为极限磁滞回线,把图1-2中原点O和各个磁滞回线的顶点 $a_1$ , $a_2$ 等连成曲线,此曲线即为铁磁材料的基本磁化曲线。不同的铁磁材料其基本磁化曲线是不相同的。根据基本磁化曲线可以近似确定铁磁材料的磁导率 $\mu$ 。从基本磁化曲线上一点到原点 O 连线,其斜率 $μ =B/H$ 定义为这种磁化状态下的磁导率。
    \subsection{测量原理}
        测量电路如图 1-3 所示,图中 A 为环状磁性材料,$N_1$、$N_2$ 分别为绕在其上的初、 次级线圈的匝数;M 为互感器;$R_0$ 是取样电阻,阻值为 1 \unit{\ohm} ;电子积分器用于测量 B,由运算放大器等构成。$K_1$ ,电流换向开关; $K_2$ ,选择测量开关;$K_3$ ,复位按钮开关;E,可调电源。
\section{实验仪器}%规格及参数
    静态磁特性参数测量仪,直流电源,计算机
\section{实验过程}%简述主要过程和实验内容
\section{实验数据}
\end{document}